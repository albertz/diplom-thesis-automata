\section{Results on concrete $*$-language classes}
\label{concrete-results}

%\subsection{Overview}
%...
In \cref{chapter:regOmegaLangs}, we already showed many results for the different $\omega$-language classes constructed based on $\Langreg$. Mainly, those are the strict inclusions as shown in the diagram in \cref{regomega-diagram}.

In \cref{general-results}, we found some general conditions on abritrary $*$-language classes $\Lang$ under which the inclusion diagram stays the same. We also gave other conditions on $\Lang$ under which the diagram becomes different or other properties change.

We will now study some concrete well-known $*$-language classes and try to apply the results from \cref{general-results} as well as study some properties individually.

%S15/S151: overview, hierarchy

% If I need syntactic semigroups, M(L), see Chapter5,p53 from FinAutLogR109.
% M(L) is finite <=> L is regular
% L in FO[\emptyset] <=> M(L) is finite, commutative and (every elem is) idempotent (p.77)
% note that FO[\emptyset] is even less than FO[=]
% more: ConcHierR104

\subsection{Starfree regular languages}
%S18: def
\label{lang:starfree}
The set of \defword{starfree $*$-languages} $\Lang(\mathtext{starfree})$ is defined as the smallest set $\Lang = \Lang(\mathtext{starfree})$ with
\begin{itemize}
\item[(a)] $\Set{a} \in \Lang$ for all $a \in \Sigma$,
\item[(b)] $\Lang$ is closed under boolean combinations,
\item[(c)] $\Lang$ is closed under concatenation.
\end{itemize}
See \cite[Section 2.2]{ConcHierR104} for further references.

\begin{lemma}
\[ \Lang(\mathtext{starfree}) \subsetneqq \Langreg \]
\begin{proof}
Obviously, all starfree $*$-languages are regular. $(\Sigma\Sigma)^*$ is a language which is not starfree (see \cite[IV.2.1]{FinAutLogR109}). Intuitively, this language is counting the letters and requires that the amount is even. Every "counting" language is not starfree.
\end{proof}
\end{lemma}

Starfree languages have a direct representing regular expression, directly derived from the definition, like $\neg(a \wedge b)$, i.e. $-(\Set{a}\cap\Set{b})$. However, they don't require that a straight-forward ($\approx$ deterministic) representing regular languages also have no stars; for the given example, that is $-\emptyset = \Sigma^*$. Thus, $\emptyset$, $\Sigma$, $\Sigma^*$, $\Sigma^+ = \Sigma^* \cdot \Sigma$, $\Set{\epsilon} = \Sigma^* - \Sigma^+$, $a^* = \Sigma^* - (\Sigma^* \cdot (\Sigma - \Set{a}) \cdot \Sigma^*)$ are all starfree languages.

In \cite[Theorem 4.10]{LangAutLogicR102}, we can see that
\[ \Lang(\mathtext{starfree}) = \Lang(\mathrm{FO[<]}) . \]
% even more: V_n = B_n in Theo 4.12
%W161,S210,S211,S101a,S51
% R107: some characterization
%In \cite[Theorem 3.2]{CombR107}, we see that
%\[  \Lang^\omega(\Set{}) ... \]

In \cite{CounterFreeMcNaughton}, the structure of automata accepting starfree languages was studied. The result was that a language $L$ is starfree exactly iff the minimal deterministic automaton for $L$ is counter-free as defined in the following.

\begin{mydef}
Let $\A = (Q,\Sigma,q_0,\delta,F)$ be an automaton. A state-sequence $s \in Q^{\ge 2}$ with $s[i] \ne s[j]$ for $i \ne j$ is a \defword{counter} iff there is a $w \in \Sigma^*$ such that
\[ \delta(s[i], w) = s[i+1] \ \ \ \forall i < \left| s \right| , \]
\[ \delta(s[\left| s \right|], w) = s[0] . \]

If $\A$ has no counter, it is is \defword{counter-free}.
\end{mydef}

First, we start with some specific study on this language class.

%S210:
\begin{theorem}
\[ \Lang^\omega (\mathrm{FO[<]}) = \BC \lim \Lang^*(\mathrm{FO[<]}) \]
\begin{proof}
Let $\varphi \in \mathrm{FO[<]}$. By the \cite[Normal Form Theorem (4.4)]{CombR107} there are bounded formulas $\varphi_1(y),\dotsb,\varphi_r(y),\psi_1(y),\dotsb,\psi_r(y)$ such that for all $\alpha \in \Sigma^\omega$:
\[ \alpha \models \varphi \Leftrightarrow \alpha \models \bigvee_{i=1}^r \left( \forall x \exists y > x \colon \varphi_i (y) \right) \wedge \neg \left( \forall x \exists y > x \colon \psi_i (y) \right) \]
Thus:
\[
\alpha \models \varphi \Leftrightarrow \bigvee_{i=1}^r
\underbrace{ \left( \alpha \models \forall x \exists y > x \colon \varphi_i (y) \right) }_
{\mathclap{\quad\quad \begin{aligned}
& \Leftrightarrow \forall x \exists y > x \colon \alpha[0,n] \models \varphi_i(\omega) \\
& \Leftrightarrow \exists^\omega n \colon \alpha[0,n] \models \varphi_i(\omega) \\
& \Leftrightarrow \alpha \in \lim L^*( \varphi_i(\omega) )
\end{aligned}}}
\wedge \neg \left( \alpha \models \forall x \exists y > x \colon \psi_i (y) \right)
\]
where $\varphi_i(\omega)$ stands for $\varphi_i$ with all bounds removed.
I.e. we have
\[ L^\omega(\varphi) = \bigcup_{i=1}^r \lim( L^* (\varphi_i (\omega)) \cap \neg \lim( L^* (\psi_i (\omega))) , \]
and thus
\[ L^\omega(\varphi) \in \BC \lim \Lang^* (\mathrm{FO[<]}) . \]
We have proved the $\subseteq$-direction. For $\supseteq$:
\begin{align*}
& \alpha \in \lim( L^*(\varphi) ) \\
\Leftrightarrow \ & \exists^\omega n \colon \alpha[0,n] \models \varphi \\
\Leftrightarrow \ & \alpha \models \forall x \exists y > x \colon \varphi(y) \\
\Leftrightarrow \ & \alpha \in L^\omega ( \forall \exists y > x \colon \varphi(y) )
\end{align*}
where $\varphi(y)$ stands for $\varphi$ with all variables bounded by $y$.
I.e.
\[ \lim \Lang^* (\mathrm{FO[<]}) \subseteq \Lang^\omega (\mathrm{FO[<]}) , \]
and thus also
\[ \BC \lim \Lang^* (\mathrm{FO[<]}) \subseteq \Lang^\omega (\mathrm{FO[<]}) . \]
Thus we have proved the equality.
\end{proof}
\end{theorem}

%S211:
\begin{theorem}
\[ \BC \ext \Lang^*( \mathrm{FO[<]} ) \subsetneqq \BC \lim \Lang^* ( \mathrm{FO[<]} ) \]
\begin{proof}
$\subseteq$: $L \subset \Sigma^\omega \text{ starfree } \ \Rightarrow\ L \Sigma^\omega \in \lim( \Lang^*( \mathrm{FO[<]} ) )$

$\neq$:
\begin{align*}
& L := (\Sigma^* a)^\omega \\
\Rightarrow \ & L = \lim( (\Sigma^* a)^* ) \\
\Rightarrow \ & L = L^\omega(\exists^\omega x : Q_a x)
\end{align*}
And we have $L \notin \BC \ext \Lang^* (\mathrm{FO[<]})$.
\end{proof}
\end{theorem}

%S101a:
$\Set{a} \in \Lang$. $a \Sigma^* \in \Lang$, thus $a\Sigma^\omega = \ext(\Set{a}) = \dext{a\Sigma^*}$. I.e. $\ext \cap \dext \Lang \ne \emptyset$.

$\Lang^*(\mathtext{starfree})$ is obviously \emph{closed under negation, suffix-independence, union and alphabet permutation}.

\begin{lemma}
\label{lang:starfree:closurefinalstates}
$\Lang^*(\mathtext{starfree})$ is \emph{closed under change of final states}.
\begin{proof}
Let $L \in \Lang^*(\mathtext{starfree})$ and let $\A_L$ be the minimal deterministic automaton accepting $L$. Then $\A_L$ is counterfree. Any change of final states of $\A_L$ keeps the counterfree property.
\end{proof}
\end{lemma}

$L_a := \Sigma^* a \in \Lang$ is an $\ext$-$\dext$-separating and $\lim$-$\dlim$-separating language.

Thus, with \cref{gen:main-theorem-inclusions}, for $\Lang := \Lang(\mathtext{starfree})$, we get
\[ \ext \cap \dext \Lang \subsetneqq
\ext \cup \dext \Lang \subsetneqq
\BC \ext \Lang =
\lim \cap \dlim \Lang \subsetneqq
\lim \cup \dlim \Lang \subsetneqq
\BC \lim \Lang . \]


\subsection{Languages of dot-depth-$n$}
\label{lang:dotdepth}
When we consider the concatination-depth of the construction of starfree languages, we get a hierarchy. The concationation-depth is also called the \defword{dot-depth}.

%S21: def
For any $n \in \N_0, r \in \N$, define recursively the set $\mathcal{B}_{n,r}$ of \emph{dot-depth $n$ languages with products of length $\le r$}:
\begin{align*}
\mathcal{B}_{0,r} & := \BC \Set{\Set{w}}{w \in \Sigma^{\le r}} \\
\mathcal{B}_{n+1,r} & := \BC \Set{B_1 \cdot \dotsc \cdot B_r}{B_i \in \mathcal{B}_{n,r}}
\end{align*}
Then, define the language class of \defword{dot-depth-$n$}
\[ \Lang(\mathtext{dot-depth-$n$}) := \bigcup_{r\in\N} \mathcal{B}_{n,r} . \]
By this definition, we get
\[ \Lang(\mathtext{starfree}) = \bigcup_{n\in\N} \Lang(\mathtext{dot-depth-$n$}) . \]
For references, see \cite{ConcatGameDotDepth}.

A known result (shown e.g. in \cite{ConcatGameDotDepth}) is the strict hierarchy
\[ \Lang(\mathtext{dot-depth-$n$}) \subsetneqq \Lang(\mathtext{dot-depth-$n+1$}). \]
Also, it was shown in \cite{RegEvTho82} that the dot-depth hierarchy characterizes exactly the \defword{first-order quantifier alternation depth} (of $\mathrm{FO[<]}$ formulas in prenex normal form) hierarchy.

For every $n\in\N_0$, $\Lang(\mathtext{dot-depth-$n$})$ is obviously closed under negation and alphabet permutation. From the construction in \cref{lang:starfree}, we see that $\Sigma^* \in \Lang(\mathtext{dot-depth-$0$})$. Thus, $L_a := \Sigma^* \cdot a \in \Lang(\mathtext{dot-depth-$1$})$. $L_a$ is a $\ext$-$\dext$-separating and $\lim$-$\dlim$-separating language. Thus, as can be seen in \cref{gen:main-theorem-inclusions}, for $n \ge 1$, $\Lang := \Lang(\mathtext{dot-depth-$n$})$, we get
\[ \ext \cap \dext \Lang \subsetneqq
\ext \cup \dext \Lang \subsetneqq
\BC \ext \Lang , \]
\[ \lim \cap \dlim \Lang \subsetneqq
\lim \cup \dlim \Lang \subsetneqq
\BC \lim \Lang . \]

For $\Lang(\mathtext{dot-depth-$0$})$, we get a different result as we see in the following.
\begin{lemma}
\label{lang:dotdepth0:ext=dext-lim=dlim}
$\Lang(\mathtext{dot-depth-$0$})$ is closed under change of final states and we have
\[ \ext \Lang(\mathtext{dot-depth-$0$}) = \dext \Lang(\mathtext{dot-depth-$0$}), \]
\[ \lim \Lang(\mathtext{dot-depth-$0$}) = \dlim \Lang(\mathtext{dot-depth-$0$}). \]
\begin{proof}
Let $w \in \Sigma^*$. Consider $L := \Set{w}$ or $L := -\Set{w}$. Let $\A = (Q,\Sigma,q_0,\delta,F)$ be a det. automaton for $L$ accepting exactly $L$. Without restriction, $\A$ has the following properties:
\begin{enumerate}
\item[(1)] $\A$ is complete, i.e. the transition path is defined for every word in $\Sigma^*$.
\item[(2)] The only loops exist in ending states $E := \Set{q \in Q}{\forall a \in \Sigma \colon \delta(q,a) = q}$.
\end{enumerate}
Then, given two automata with such properties, the product automata $\Ax$ also has these properties. Thus, all boolean combinations of such automata keep those properties. That are all possible automata for $\Lang(\mathtext{dot-depth-$0$})$ languages.

Now, let $\A = (Q,\Sigma,q_0,\delta,F)$ be any such automata accepting exactly a language $L \in \Lang(\mathtext{dot-depth-$0$})$. $\ext \Lang \subseteq \dext \Lang$ means that we can transform the E-acceptance into A-acceptance. We get this by
\[ F_{E\rightarrow A} := F \cup \Set{q \in Q}{\text{there is a way from $q$ to $F$}} \cup \delta(F,\Sigma^*) . \]
Because of the property (2) of $\A$, this will accept the same languages. For the other way around, we get
\[ F_{A\rightarrow E} := F \cap E . \]
All modifications of the final state set stays in the same language class, given the above properties. Thus, we have
\[ \ext \Lang(\mathtext{dot-depth-$0$}) = \dext \Lang(\mathtext{dot-depth-$0$}) . \]

For Büchi/co-Büchi, we don't need to change the final state set at all because all loops are only in the ending states, thus we always have $\lim L = \dlim L$.
\end{proof}
\end{lemma}

And also:
\begin{lemma}
\[ \BC \ext \Lang(\mathtext{dot-depth-$0$}) = \lim \cap \dlim \Lang(\mathtext{dot-depth-$0$}) \]
\begin{proof}
Define $\Lang := \Lang(\mathtext{dot-depth-$0$})$.

Let $\A$ be an A-automaton with $L^*(\A) \in \Lang$ with the properties as in \cref{lang:dotdepth0:ext=dext-lim=dlim}. Because $\dext \Lang = \BC \ext \Lang$, $L^\omega_A(\A)$ is representing any language from $\BC \ext \Lang$. Then,
\[ L^\omega_A(\A) = L^\omega_\text{Büchi}(\A) = L^\omega_\text{co-Büchi}(\A) . \]
I.e.
\[ \BC \ext \Lang \subseteq \lim \cap \dlim \Lang . \]
From that, we also see that
\[ \lim \cup \dlim \Lang \subseteq \dext \Lang , \]
i.e. esp.
\[ \lim \cap \dlim \Lang \subseteq \BC \ext \Lang. \]
%Given two automata $\A_1$, $\A_2$ with $L^*(\A_1), L^*(\A_2) \in \Lang$ with the properties as in \cref{lang:dotdepth0:ext=dext,lim=dlim} and $L^\omega_\text{Büchi}(\A_1) = L^\omega_\text{co-Büchi}(\A_2)$.
\end{proof}
\end{lemma}

Thus, for $\Lang := \Lang(\mathtext{dot-depth-$0$})$ we have
\[ \ext \cap \dext \Lang =
\ext \cup \dext \Lang =
\BC \ext \Lang =
\lim \cap \dlim \Lang =
\lim \cup \dlim \Lang =
\BC \lim \Lang . \]

\subsection{Piecewise testable languages}
%S19: def
\label{lang:PT}
Let $u,v \in \Sigma^*$. For $n \in \N$, define the congruence relation $\simeq_{\PT_n}$:
\[ u \simeq_{\PT_n} v \ \ \ :\Leftrightarrow \ \ \ \operatorname{Subwords}_{\le n}(u) = \operatorname{Subwords}_{\le n}(v) , \]
where
\[ \operatorname{Subwords}_{\le n}(u) := \Set{w \in \Sigma^{\le n}}{\text{$w$ is a subword of $u$}} \]
and $w$ is a \defword{subword} of $u$ for $w,u \in \Sigma^*$ iff there is a subsequence $(s_n)_{n\in\N}$ of $(n)_{n\in\N}$ such that $w = u|_s$.

Define
\[ \Lang(\PT_n) := \Lang(\simeq_{\PT_n}). \]
Then the class of \defword{piecewise testable languages} is defined as
\[ \Lang(\PT) := \bigcup_{n\in\N} \Lang(\PT_n) . \]
We also have the characterization
\[ \Lang(\PT) = \BC \bigcup_{n\in\N_0,a_i \in \Sigma} \Set{\Sigma^* a_1 \Sigma^* a_2 \Sigma^* \cdots\Sigma^* a_n \Sigma^*} . \]
% The syntactic monoid is also J-trivial.
See \cite[Section 2.3]{ConcHierR104} for further references.

\begin{lemma}
\[\Lang(\PT) \subsetneqq \Lang(\mathtext{starfree})\]
\begin{proof}
Let $n\in\N$, $a_i \in \Sigma$. $\Sigma^*$ and $\Set{a_i}$ are starfree. Thus, $\Sigma^* a_1 \Sigma^* a_2 \Sigma^* \cdots\Sigma^* a_n \Sigma^*$ is starfree. And starfree languages are closed under boolean operations. This proofs the inclusion.

$\Sigma^* aa \Sigma^*$ is a starfree language which is not piecewise testable.
\end{proof}
\end{lemma}

%S52: overview
%S212: BC lim class

%S101:
\begin{theorem}
\label{thm.PT}
\[ \BC \ext \Lang^* (\PT) = \BC \lim \Lang^* (\PT) \]
\begin{proof}
$\subseteq$: It is sufficient to show $\ext(\Lang^* (\PT)) \subseteq \BC \lim \Lang^* (\PT)$.

By complete induction:
\begin{align*}
& \ext(\Sigma^* a_1 \Sigma^* a_2 \dotsb a_n \Sigma^*) = \Sigma^* a_1 \Sigma^* a_2 \dotsb a_n \Sigma^\omega = \lim(\Sigma^* a_1 \Sigma^* a_2 \dotsb a_n \Sigma^*) \\
& \ext( \neg ( \Sigma^* a_1 \Sigma^* a_2 \dotsb a_n \Sigma^* )) = \Sigma^\omega = \lim( \Sigma^* ) \\
& \ext( \emptyset ) = \emptyset = \lim(\emptyset)
\end{align*}

It is sufficient to show negation only for such ground terms because we can always push the negation down.
\begin{align*}
& \ext(A \cup B) = \ext(A) \cup \ext(B)
\end{align*}

This makes the induction complete.

$\supseteq$: It is sufficient to show $\lim(\Lang^* (\PT)) \subseteq \BC \ext \Lang^* (\PT)$.
\begin{align*}
& \lim(\emptyset) = \ext(\emptyset), \ \lim(\Sigma^* a_1 \Sigma^* a_2 \dotsb a_n \Sigma^*) = \ext(\Sigma^* a_1 \Sigma^* a_2 \dotsb a_n \Sigma^*) \ \ \text{(see above)} \\
& \begin{aligned}
\lim( \neg ( \Sigma^* a_1 \Sigma^* a_2 \dotsb a_n \Sigma^* ) ) & = \Set{ \alpha \in \Sigma^\omega }{ \exists^\omega n \colon \alpha [0,n] \notin \Sigma^* a_1 \Sigma^* a_2 \dotsb a_n \Sigma^* } \\
& = \Set{ \alpha \in \Sigma^\omega }{ \forall n \colon \alpha [0,n] \notin \Sigma^* a_1 \Sigma^* a_2 \dotsb a_n \Sigma^* } \\
& = \neg \ext( \Sigma^* a_1 \Sigma^* a_2 \dotsb a_n \Sigma^* )
\end{aligned} \\
& \begin{aligned}
\lim(A \cup B) & = \Set{ \alpha \in \Sigma^\omega }{ \exists^\omega n \colon \alpha[0,n] \in A \cup B } = \lim(A) \cup \lim(B) \\
\lim(A \cap B) & = \Set{ \alpha \in \Sigma^\omega }{ \exists^\omega n \colon \alpha[0,n] \in A \cap B } \\
\intertext{and because $A,B$ are piece-wise testable}
& = \Set{ \alpha \in \Sigma^\omega }{ \exists n : \forall m > n \colon \alpha[0,m] \in A \cap B } = \lim(A) \cap \lim(B)
\end{aligned}
\end{align*}
\end{proof}
\end{theorem}

$\Lang(\PT_n)$ is defined via the congruence relation $\simeq_{\text{PT}_n}$. The set of equivalence classes is finite. And we get a canonical $\simeq_{\text{PT}_n}$-automaton which we will call just $\PT_n$-automaton. The state set of such automaton is $S_{\PT_n} := \Sigma^* / \simeq_{\text{PT}_n}$.

Thus, by \cref{gen:R:closurelemma}, $\Lang(\PT_n)$ is closed under negation, union, intersection and change of final states.

\begin{lemma}
$\Lang(\PT_n)$ is closed under suffix-independence.
\begin{proof}
Look at some $\PT_n$-automaton $\A$ with final state set $F \subseteq S_{\PT_n}$. Any state in $S_{\PT_n}$ describes what letters $M \subseteq \Sigma$ we have seen so far. This can only increase. Thus, there is no way back. Let $F' \subseteq S_{\PT_n}$ be all states from $F$ and all that follow. Then
\[ L^*(\A(F')) \cdot \Sigma^* = L^*(\A(F)) . \]
\end{proof}
\end{lemma}

Obviously, $\Lang(\PT_n)$ is also closed under alphabet permutation. And $L_a := \Sigma^* a \Sigma^* \in \Lang(\PT_n)$ is a $\ext$-$\dext$-separating language.

Thus, via \cref{gen:main-theorem-inclusions}, for $\Lang := \Lang(\PT_n)$, we have
\[ \ext \cap \dext \Lang \subsetneqq
\ext \cup \dext \Lang \subsetneqq
\BC \ext \Lang =
\lim \cap \dlim \Lang \subseteq
\lim \cup \dlim \Lang \subseteq
\BC \lim \Lang . \]

As we have said, in the $\PT_n$-automaton, there are never transitions back. Thus, all loops in the $\PT_n$-automaton only loop in a single state. Thus,
\[ \lim \Lang(\PT_n) = \dlim \Lang(\PT_n) . \]
It follows
\[ \lim \cap \dlim \Lang(\PT_n) =
\lim \cup \dlim \Lang(\PT_n) =
\BC \lim \Lang(\PT_n) . \]

Via \cref{gen:R-bclim-cap-ext}, we have
\[ \BC \lim \Lang(\PT_n) \cap \ext \Langreg = \ext \Lang(\PT_n) . \]
$\Lang(\PT_n)$ is obviously also not \emph{postfix-loop-deterministic}. Thus, with \cref{gen:postfixloopdet<-BclimAndLreg=lim},
\[ \BC \lim \Lang(\PT_n) \cap \lim \Langreg = \lim \Lang(\PT_n) . \]

For $\Lang(\PT)$, via \cref{gen:extInBCext} and $L_a$, we also have
\[ \ext \cap \dext \Lang(\PT) \subsetneqq
\ext \cup \dext \Lang(\PT) \subsetneqq
\BC \ext \Lang(\PT) . \]

We also have $\Lang(\PT_n) \subseteq \Lang(\PT_{n+1})$ for all $n \in \N$. Thus, we have $\operatorname{op} \Lang(\PT_n) \subseteq \operatorname{op} \Lang(\PT_{n+1})$ for $\operatorname{op} \in \Set{\ext, \dext, \BC \ext, \lim,\dlim,\BC \lim}$.

From the results so far, for $\Lang := \Lang(\PT)$, it follows
\[ \ext \cap \dext \Lang \subsetneqq
\ext \cup \dext \Lang \subsetneqq
\BC \ext \Lang =
\lim \cap \dlim \Lang =
\lim \cup \dlim \Lang =
\BC \lim \Lang . \]

To extend that, we note that every loop in a $\PT_n$-automaton is looping only in a single state because we never can get back as it was noted earlier. Thus, by \cref{gen:infPostfixIndep-AutLoops}, $\Lang(\PT_n)$ is \emph{infinity-postfix-independent} (\cref{gen:def:infinity-postfix-independent}). Also, $\Lang(\PT_n)$ is not \emph{postfix-loop-deterministic} (\cref{gen:def:postfix-loop-deterministic}).

By \cref{gen:bcLimLReg-and-limL} or \cref{gen:postfix-loop-det=limAndBclim}, we get
\[ \BC \lim \Lang(\PT_n) \cap \lim \Langreg = \lim \Lang(\PT_n) . \]
I.e. we also have
\[ \BC \lim \Lang(\PT) \cap \lim \Langreg = \lim \Lang(\PT) . \]

\

\label{lang:posPT}
\defword{Positive piece-wise testable languages} are defined as
\[ \Lang(\posPT) := \posBC \bigcup_{n\in\N_0,a_i \in \Sigma} \Set{\Sigma^* a_1 \Sigma^* a_2 \Sigma^* \cdots\Sigma^* a_n \Sigma^*} , \]
where $\posBC$ is defined like $\BC$ except the negation.

For positive piece-wise testable (pos-PT) languages, we get the same result on the $\BC \ext$ and $\BC \lim$ relation.
%\subsection{positive piece-wise testable}
%S53: overview
%S214: BC lim class

%S101:
\begin{theorem}
\label{thm.ext.lim.posPT}
\[ \BC \ext \Lang^* (\posPT) = \BC \lim \Lang^* (\posPT) \]
\begin{proof}
$\subseteq$: Exactly like the proof for \cref{thm.PT} except that we leave out the negated part.

$\supseteq$: Also like the proof for \cref{thm.PT}.
\end{proof}
\end{theorem}

We note that we cannot make good reasonings about $\dext$ and $\dlim$ because we don't have the negation closure. However, we can define
\[ \neg\ext \Lang := \Set{-\ext L}{L \in \Lang} , \]
\[ \neg\lim \Lang := \Set{-\lim L}{L \in \Lang} . \]
When using $\neg\ext$ and $\neg\lim$ instead of $\dext$ and $\dlim$, we get the extended result
\[ \BC \ext \Lang(\mathtext{pos-PT}) =
\lim \cap \neg\lim \Lang(\posPT) =
\lim \cup \neg\lim \Lang(\posPT) =
\BC \lim \Lang(\posPT) .\]

Also, the $L_a$ from above is also in $\Lang(\posPT)$. When using $\neg\ext$, we don't need the negation closure to use \cref{gen:extInBCext}. Thus, we have
\[ \ext \cap \neg\ext \Lang(\posPT) \subsetneqq
\ext \cup \neg\ext \Lang(\posPT) \subsetneqq
\BC \ext \Lang(\posPT) . \]

We also have a relation between pos-PT and PT.

\begin{lemma}
\[ \BC \ext \Lang^*(\mathtext{pos-PT}) = \BC \ext \Lang^* (\mathtext{PT}) \]

\begin{proof}
In the proof of $\lim \Lang^*(\mathtext{PT}) \subseteq \BC \ext \Lang^* (\mathtext{PT})$ we actually proved $\BC \lim \Lang^*(\mathtext{PT}) \subseteq \BC \ext \Lang^* (\mathtext{pos-PT})$. Similiarly we also proved $\BC \ext \Lang^* (\mathtext{PT}) \subseteq \BC \lim \Lang^* (\mathtext{pos-PT})$.

With \cref{thm.PT} and \cref{thm.ext.lim.posPT} we get the claimed equality.
\end{proof}
\end{lemma}

\subsection{Locally testable languages}
\label{lang:LT}
%S16: def
Let $u,v \in \Sigma^*$. For $n \in \N$, define the congruence relation $\simeq_{\text{LT}_n}$:
\begin{alignat*}{3}
u \simeq_{\text{LT}_n} v \ \ \ && :\Leftrightarrow \ \ \ & \operatorname{left-Fact}_{<n}(u) = \operatorname{left-Fact}_{<n}(v) , \\
&&& \operatorname{right-Fact}_{<n}(u) = \operatorname{right-Fact}_{<n}(v) , \\
&&& \operatorname{Fact}_{n}(u) = \operatorname{Fact}_{n}(v) \\
&& \Leftrightarrow \ \ \ & u \simeq_{\text{endwise}_n} v , \\
&&& \operatorname{Fact}_n(u) = \operatorname{Fact}_n(v)
\end{alignat*}
where
\begin{align*}
\operatorname{left-Fact}_{<n}(v) & := \Set{w \in \Sigma^{<n}}{\text{$w$ is left-factor of $v$}} \\
\operatorname{right-Fact}_{<n}(v) & := \Set{w \in \Sigma^{<n}}{\text{$w$ is right-factor of $v$}} \\
\operatorname{Fact}_{n}(v) & := \Set{w \in \Sigma^{n}}{\text{$w$ is factor of $v$}}.
\end{align*}
For $v,w \in \Sigma^*$,
\begin{align*}
\text{$w$ is a \defword{left-factor} of $v$} & \ :\Leftrightarrow \ \exists u \in \Sigma^* \colon wu = v \\
\text{$w$ is a \defword{right-factor} of $v$} & \ :\Leftrightarrow \ \exists u \in \Sigma^* \colon uw = v \\
\text{$w$ is a \defword{factor} of $v$} & \ :\Leftrightarrow \ \exists u_1,u_2 \in \Sigma^* \colon u_1 w u_2 = v.
\end{align*}

Define
\[ \Lang(\LT_n) := \Lang(\simeq_{\text{LT}_n}) . \]
Then the class of \defword{locally testable languages} is defined as
\[ \Lang(\LT) := \bigcup_{n\in\N} \Lang(\LT_n) . \]
We also have the characterization
\[ \Lang(\LT) = \BC \bigcup_{u,v,w \in \Sigma^+} \Set{u\Sigma^*, \Sigma^* v, \Sigma^* w \Sigma^*, \Set{\epsilon}} . \]
See \cite[Section 2.5]{ConcHierR104} for further references.

We also have more direct connection to starfree languages:
\[ \Lang(\LT) = \Lang(\mathtext{dot-depth-$1$}) \]
See \cite{CharVarietiesKnast} for references.

\begin{theorem}
\[ \BC \ext \Lang^* (\LT) \subsetneqq \BC \lim \Lang^*(\LT)  \]

\begin{proof}
Let $w \in \Sigma^+$.
\begin{align*}
& \ext (w \Sigma^*) = \lim(w \Sigma^*) \\
& \ext( \Sigma^* w) = \Sigma^* w \Sigma^\omega = \lim( \Sigma^* w \Sigma^* ) \\
& \ext( \Sigma^* w \Sigma^*) = \Sigma^* w \Sigma^\omega = \lim( \Sigma^* w \Sigma^* )
\end{align*}
Thus we have
\[ \BC \ext \Lang^* (\LT) \subseteq \BC \lim \Lang^*(\LT) . \]
But we also have
\[ \lim(\Sigma^*) = (\Sigma^* w)^\omega \notin \BC \ext \Lang^* (\LT) . \]
\end{proof}
\end{theorem}

$\Lang(\LT_n)$ is defined via the congruence relation $\simeq_{\text{LT}_n}$. The set of equivalence classes is finite. And we get a canonical $\simeq_{\text{LT}_n}$-automaton which we will call just $\LT_n$-automaton.

Thus, by \cref{gen:R:closurelemma}, $\Lang(\LT_n)$ is closed under negation, union, intersection and change of final states.

% TODO... what about suffix-independence closure?

About the relation to $\Langreg$: Our tools are \cref{gen:bcLimLReg-and-limL} or \cref{gen:postfix-loop-det=limAndBclim}.
\begin{lemma}
Let $a,b \in \Sigma$. $\Lang(\LT_n)$ is \emph{postfix-loop-deterministic} (\cref{gen:def:postfix-loop-deterministic}) and not \emph{infinity-postfix-independent} (\cref{gen:def:infinity-postfix-independent}) for $n \ge 2$.
\begin{proof}
Let $u \in \Sigma^*$ so that every word in $\Sigma^2$ is a factor of $u$. Then, in the $LT_2$-automata, we are in a final SCC $S$ where we cannot get out anymore because all left-factors and all factors are fixed. We can only change the right-factors (of len $<2$). The states of $S$ are
\[ S = \Set{\left< wa \right>, \left< wb \right>} . \]
And we have two loops
\[ P_1 : \left< wa \right> \xrightarrow{a} \left< wa \right> \]
and
\[ P_2 : \left< wb \right> \xrightarrow{b} \left< wb \right> , \]
where $P_1,P_2 \subseteq S$, $P_1 \ne P_2$ and $P_1 \not\subseteq P_2$, $P_2 \not\subseteq P_1$. I.e. $\Lang(\LT_2)$ is postfix-loop-deterministic and not infinty-postfix-independent.
This can be extended in a similar way for any $n > 2$.
\end{proof}
\end{lemma}

We cannot apply \cref{gen:bcLimLReg-and-limL}.  But by \cref{gen:postfix-loop-det=limAndBclim}, for all $n \ge 2$, we have
\[ \BC \lim \Lang(\LT_n) \cap \lim \Langreg \supsetneqq \lim \Lang(\LT_n) . \]
Unfortunately, we cannot argue directly about $\Lang(\LT)$ with this result.

For $LT_1$, it looks different. The right-factors of len $<1$ don't tell anything anymore about a word. Thus, $\Lang(\LT_1)$ is not postfix-loop-deterministic. I.e., by \cref{gen:bcLimLReg-and-limL}, we have
\[ \BC \lim \Lang(\LT_1) \cap \lim \Langreg = \lim \Lang(\LT_1) . \]

%\emph{postfix-loop-deterministic} (\cref{gen:def:postfix-loop-deterministic})

% TODO: I think R105... handles them?
%\defword{Positive locally testable languages} are defined as
%\[ \Lang(\mathtext{pos-LT}) = \operatorname{pos-BC} \bigcup_{u,v,w \in \Sigma^+} \Set{u\Sigma^*, \Sigma^* v, \Sigma^* w \Sigma^*, \Set{\epsilon}} . \]

\subsection{Locally threshold testable languages}
\label{lang:LTT}
%S17: def
Let $u,v \in \Sigma^*$. Let $k,r \in \N$, define the congruence relation $\simeq_{\LTT^k_r}$:
\begin{alignat*}{3}
u \simeq_{\LTT^k_r} v \ \ \ && :\Leftrightarrow \ \ \ & \operatorname{left-Fact}_{<n}(u) = \operatorname{left-Fact}_{<n}(v) , \\
&&& \operatorname{right-Fact}_{<n}(u) = \operatorname{right-Fact}_{<n}(v) , \\
&&& \begin{aligned}
\forall x \in \Sigma^{\le k} \colon & \operatorname{count-Fact}_x(u) = \operatorname{count-Fact}_x(v) < r \ \ \ \ \text{or} \\
& \min \Set{\operatorname{count-Fact}_x(u), \operatorname{count-Fact}_x(v)} \ge r
\end{aligned}
\end{alignat*}
where for $x \in \Sigma^*$, $\operatorname{count-Fact}_x(u)$ states how often $x$ occurs as a factor in $u$, formally
\[ \operatorname{count-Fact}_x(u) := \#\Set{n \in \N}{u[n,n+\vert x \vert - 1] = x} . \]
Define
\[ \Lang(\LTT^k_r) := \Lang(\simeq_{\LTT^k_r}). \]
Then the class of \defword{locally threshold testable languages} is defined as
\[ \Lang(\LTT) := \bigcup_{k,r\in\N} \Lang(\LTT^k_r) . \]

Locally threshold testable languages are a generalization of locally testable language. We can see that
\[  \Lang(\LT) = \bigcup_{k\in\N} \Lang(\LTT^k_1) . \]

For further references, see \cite[IV.3]{FinAutLogR109}.

In \cite[IV.3.3]{FinAutLogR109} or \cite[Corollary 4.9]{LangAutLogicR102}, we can see that
\[  \Lang(\LTT) = \Lang(\mathrm{FO[+1]}). \]
In \cite[IV.3.4]{FinAutLogR109}, we see that
\[  \Lang(\LTT) \subsetneqq \Lang(\mathtext{starfree}). \]

% FinAutLogR109, p.51:
% L in FO[=] <=> L is union of LTT^1_r classes for some r

%\subsection{FO[+1]}
%def: S17
%S213: BC ext class
%215: BC lim class

%S103:
\begin{theorem}
\[ \Lang^\omega (\mathrm{FO[+1]}) = \BC \ext \Lang^*(\mathrm{FO[+1]}) \]
\begin{proof}
From \cite[Theorem 4.8]{LangAutLogicR102}, we know that each formular in FO[+1] is equivalent (for both finite and infinite words) to a boolean combination of statements ``sphere $\sigma \in \Sigma^+$ occurs $\geq n$ times''. That statement can be expressed by a sentence of the form
\[ \psi := \exists \overline{x_1} \dotsb \exists \overline{x_n} \varphi(\overline{x_1}, \dotsb, \overline{x_n}) \]
where each $\overline{x_i}$ is a $\abs{\sigma}$-tuple of variables and the formula $\varphi$ states:
\[
\bigwedge_{\mathclap{i,j \in \underline{n},\atop { i \neq j,\atop k,l \in \underline{\abs{\sigma}}}}} x_{i,k} \neq x_{j,l}
\ \wedge\ \bigwedge_{\mathclap{i \in \underline{n},\atop k \in \underline{\abs{\sigma}-1}}} x_{i,k+1} = x_{i,k} + 1
\ \wedge\ \bigwedge_{\mathclap{i \in \underline{n},\atop k \in \underline{\abs{\sigma}}}} Q_{\sigma_k} x_{i,k}
\]
For $\psi$, we have:
\[ \alpha \models \psi \Leftrightarrow \exists n \colon \alpha[0,n] \models \psi \ \ \text{for all } \alpha \in \Sigma^\omega , \]
i.e.
\[ L^\omega(\psi) = \ext L^*(\psi) . \]
Any formular in FO[+1] can be expressed as a boolean combination of $\psi$-like formular. With
\begin{align*}
& L^\omega(\neg \psi) = \neg L^\omega(\psi) \\
& L^\omega(\psi_1 \wedge \psi_2) = L^\omega(\psi_1) \cap L^\omega(\psi_2) \\
& L^\omega(\psi_1 \vee \psi_2) = L^\omega(\psi_1) \cup L^\omega(\psi_2)
\end{align*}
we get:
\[ \Lang^\omega(\mathrm{FO[+1]}) = \BC \ext \Lang^*(\mathrm{FO[+1]}). \]
\end{proof}
\end{theorem}

%\subsection{FO[]}

\subsection{Endwise testable languages}
\label{lang:endwise}
%S20: def
Let $u,v \in \Sigma^*$. For $n \in \N$, define the congruence relation $\simeq_{\text{endwise}_n}$:
\begin{alignat*}{3}
u \simeq_{\text{endwise}_n} v \ \ \ && :\Leftrightarrow \ \ \ & \operatorname{left-Fact}_{<n}(u) = \operatorname{left-Fact}_{<n}(v) , \\
&&& \operatorname{right-Fact}_{<n}(u) = \operatorname{right-Fact}_{<n}(v)
\end{alignat*}
Then the class of \defword{endwise testable languages} is defined as
\[ \Lang(\mathtext{endwise}) := \bigcup_{n\in\N} \Lang(\simeq_{\text{endwise}_n}) . \]

We also have the characterization
\[ \Lang(\mathtext{endwise}) = \Set{X \Sigma^* Y \cup Z}{X,Y \subseteq \Sigma^+, Z \subseteq \Sigma^*, \text{$X,Y,Z$ finite}} . \]

See \cite[Section 2.4]{ConcHierR104} for further references.

%S104:
\begin{itemize}
\item $\BC \ext \Lang^*(\mathtext{endwise}) \neq \BC \lim \Lang^*(\mathtext{endwise})$ because $\Sigma^*a \in \Lang^*(\mathtext{endwise})$.
\item $\ext(a\Sigma^* a) = a \Sigma^* a \Sigma^\omega \notin \BC \lim \Lang^*(\mathtext{endwise})$
\end{itemize}

\subsection{Finite / Co-finite languages}
%S22: def
\label{lang:finite}

The class of finite $*$-languages is denoted by $\Lang(\mathtext{finite})$. The class of infinite $*$-languages is denoted by $\Lang(\mathtext{co-finite})$. Formally

We get the following results:
%S105:
\begin{itemize}
\item $\ext \Lang(\mathtext{finite}) = \Lang(\mathtext{finite}) \cdot \Sigma^\omega$
\item $\dext \Lang(\mathtext{finite}) = \Set{\emptyset}$
\item $\neg\ext \Lang(\mathtext{finite}) = \Set{-\ext L}{L \in \Lang(\mathtext{finite})} = \Set{\dext L}{L \in \Lang(\mathtext{co-finite})} = \dext \Lang(\mathtext{co-finite})$
\item $\lim \Lang(\mathtext{finite}) = \Set{\emptyset}$
\item $\dlim \Lang(\mathtext{finite}) = \Set{\emptyset}$
\item $\neg\lim \Lang(\mathtext{finite}) = \Set{\Sigma^\omega}$
\item $\BC \lim \Lang(\mathtext{finite}) = \Set{\emptyset,\Sigma^\omega}$
\item $\ext \Lang(\mathtext{co-finite}) = \Set{\Sigma^\omega}$
\item $\neg\ext \Lang(\mathtext{co-finite}) = \Set{\emptyset}$
\item $\BC \ext \Lang(\mathtext{co-finite}) = \Set{\emptyset,\Sigma^\omega}$
\item $\lim \Lang(\mathtext{co-finite}) = \Set{\Sigma^\omega}$
\item $\dlim \Lang(\mathtext{co-finite}) = \Set{\Sigma^\omega}$
\item $\neg\lim \Lang(\mathtext{co-finite}) = \dlim \Lang(\mathtext{finite}) = \Set{\emptyset}$
\item $\BC \lim \Lang(\mathtext{co-finite}) = \Set{\emptyset, \Sigma^\omega}$
\end{itemize}

Thus, the usual inclusions don't hold at all here. Esp. we again have
\[ \BC \ext \Lang(\mathtext{finite}) \supsetneqq \BC \lim \Lang(\mathtext{finite}) , \ \ \ 
\BC \ext \Lang(\mathtext{co-finite}) = \BC \lim \Lang(\mathtext{co-finite}) . \]
We also have
\[ \BC \ext \Lang(\mathtext{co-finite}) =
\BC \lim \Lang(\mathtext{finite}) = \BC \lim \Lang(\mathtext{co-finite}) . \]

%\subsection{Local languages}
%\label{lang:local}
%S22: def
%A language $L \subseteq \Sigma^*$ is called \defword{local} iff there exist $P,S \subseteq \Sigma$, $N \subseteq \Sigma^2$ such that
%\[ L - \Set{\epsilon} = (P\Sigma^* \cap \Sigma^* S) - (\Sigma^* N \Sigma^*). \]
%For references, see \cite{Berstel1996439}.

%\subsection{$L$-trivial languages}
%\label{lang:Ltrivial}

%\subsection{$R$-trivial languages}
%\label{lang:Rtrivial}
%S23: def

%\subsection{Locally modulo testable languages}
%\label{lang:LmodT}
%S24: def

%\subsection{Context free languages}
%\label{lang:contextfree}
