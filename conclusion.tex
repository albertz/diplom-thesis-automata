%!TEX root =  index.tex
\section{Conclusion}
\label{chapter:conclusion}

Via language class operators like $\ext$ or $\lim$, we induced $\omega$-language classes from $*$-language classes. We presented the known results on the relations of the induced $\omega$-language classes from $\Langreg$. Those relations were shown in the diagram in \cref{regomega-diagram}.

Then, we worked on several generic cases on arbitrary $*$-language classes $\Lang$ such that we get the expected inclusions as in the diagram or at least some of them. However, we have also seen several examples where we have different relations. We introduced some closure properties on $\Lang$ which lead to the expected results. The closure under negation, union or intersection is given for most $*$-language classes. Important for some relations, e.g. the $\BC \ext \Lang = \lim \cap \dlim \Lang$ is the closure under change of final states. This property had to be adjusted several times to be powerful enough so that the proofs worked and that it still applied to important language classes such as starfree languages or all the congruence based language classes.

In \cref{gen:section:kleene}, we studied the relation between $\BC \lim \Lang$ and $\Kleene \Lang$. We saw that the equality as for $\Langreg$ didn't easily followed in the generic case with any of the given introduced closure properties on $\Lang$. Some much more strong closure of change of final state in any deterministic automata was needed. Thus, finding better properties on $\Lang$ for when the equality or inclusion in some direction holds is open for future research.

This suggests that the closure of change of final state should be separated into several different closure properties. Or there is a better different way to characterize some similar property. Also, maybe a completely different approach might be useful because many of the current proofs depend on some automaton construction or modification. This restricts the proof on subclasses of the class of regular languages.

In \cref{gen:R}, we have seen that congruence based language classes gives us many nice properties which makes arguing about the induced $\omega$-languages classes easy. However, as many important language classes are not based on congruences, namely the starfree languages or the whole class of regular languages, we cannot directly apply the results in many cases. Even in the case for locally testable languages or piecewise testable languages, we can only directly apply the results for the given $\LT_n$ or $\PT_n$ relation but not for the whole class $\LT$ or $\PT$. A possible generalization for the section would be if we have a class of $\Lang$-automata instead of a single $\Lang$-automata. In the case of $\PT$, this would be $\bigcup_n \text{$\PT_n$-automata}$. In \cref{lang:PT}, we actually see this applied.

In \cref{concrete-results}, we studied some of the most well-known subclasses of the class of regular $*$-languages. There are many more regular language subclasses like local languages, $L$-trivial or $R$-trivial languages or locally modulo testable languages. On all these classes, we can probably apply some of the results from \cref{general-results}.

Also interesting are supersets of the class of regular $*$-languages. One important class are the context-free languages. Also deterministic context-free languages or context-sensitive languages might be interesting to study in this context. A starting point to study these classes are to extend the results from \cref{general-results} to pushdown automata or other more generic cases.

Also interesting are further variations of the presented language classes, such as their only-positive variants (like $\posPT$).

Overall, we developed some powerful theorems but many questions or ideas remain open for future work.
