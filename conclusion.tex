%!TEX root =  index.tex
\section{Conclusion}
\label{chapter:conclusion}

TODO: Rückblick

In \cref{gen:R}, we have seen that congruence based language classes gives us many nice properties which makes arguing about the induced $\omega$-languages classes easy. However, as many important language classes are not based on congruences, namely the starfree languages or the whole class of regular languages, we cannot directly apply the results in many cases. Even in the case for locally testable languages or piecewise testable languages, we can only directly apply the results for the given $\LT_n$ or $\PT_n$ relation but not for the whole class $\LT$ or $\PT$. A possible generalization for the section would be if we have a class of $\Lang$-automata instead of a single $\Lang$-automata. In the case of $\PT$, this would be $\bigcup_n \text{$\PT_n$-automata}$. In \cref{lang:PT}, we actually see this applied.

In \cref{concrete-results}, we studied some of the most well-known subclasses of the class of regular $*$-languages. There are many more regular language subclasses like local languages, $L$-trivial or $R$-trivial languages or locally modulo testable languages. On all these classes, we can probably apply some of the results from \cref{general-results}.

Also interesting are supersets of the class of regular $*$-languages. One important class are the context-free languages. Also deterministic context-free languages or context-sensitive languages might be interesting to study in this context. A starting point to study these classes are to extend the results from \cref{general-results} to pushdown automata or other more generic cases.
