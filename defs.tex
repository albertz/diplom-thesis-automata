%!TEX root =  index.tex

% Quelle: Automata, Semigroups, Logic and Games - Pin, Perrin

\section{Automaton}

An \defword{automaton} $\A$ on the alphabet $\Sigma$ is given by a set $Q$ of states and a subset $E \subset Q \times A \times Q$ of transitions. In most cases we also have a subset $I \subset Q$ of initial states and a subset $F \subset Q$ of final states.

We write:
\[ \A = (Q, \Sigma, E, I, F). \]

The automaton is \defword{finite} iff $Q$ and $\Sigma$ are finite.

The automaton is \defword{deterministic} iff $E$ is a set of functions $Q \times A \rightarrow \Q$ and there is only a single initial state.

\subsection{Run}
Two transitions $(p,a,q), (p',a',q') \in E$ are \defword{consecutive} iff $q=p'$.

A \defword{run} in the automaton $\A$ is a sequence of consecutive transitions, written as:
\[ q_0 \rightarrow^{a_0} q_1 \rightarrow^{a_1} q_2 \dots \]

\subsection{Acceptence of finite words}

An automaton $\A = (Q, \Sigma, E, I, F)$ \defword{accepts} a finite word $w = (a_0,a_1,...,a_n) \in \Sigma^*$ iff there is a run $q_0 \rightarrow^{a_0} q_1 \rightarrow^{a_1} q_2 \dots \rightarrow^{a_n} q_{n+1}$ with $q_0 \in I$ und $q_{n+1} \in F$.

The language $L^*(\A)$ is defined as set of all words which are accepted by $\A$.

\section{$*$-languages}
The $*$-languages are all languages of words $w \in \Sigma^*$, i.e. the set of languages of finite words.

\subsection{regular languages}
A languages is \defword{regular} iff an automaton accepts it.

\subsection{piece-wise testable}
\subsection{$k$-locally testable}
\subsection{dot-depth-$n$}
\subsection{starfree}
\subsection{locally modulo testable}
\subsection{$R$-trivial}
\subsection{endlich / co-endlich}
\subsection{endwise testable}

\section{$\omega$-languages}
\subsection{Büchi automaton}
An automaton $\A = (Q, \Sigma, E, I, F)$ \defword{Büchi-accepts} a word $\alpha = (a_0,a_1,a_2,...) \in \Sigma^\omega$ iff there is an infinite run $q_0 \rightarrow^{a_0} q_1 \rightarrow^{a_1} q_2 \rightarrow^{a_2} q_3 ...$ with $q_0 \in I$ and $\{ q_i | q_i \in F \}$ infinite, i.e. which reaches a state in $F$ infinitely often.

The language $L^\omega(\A)$ is defined as the set of all infinite words which are Büchi-accepted by  $\A$.

An automaton $\A$ is a Büchi automaton iff we use the Büchi-acceptence.

\subsection{Muller automaton}
A Muller automaton $\A$ is a finite, deterministic automaton with \defword{Muller acceptence} and a set $\T \in 2^Q$, called the \defword{table} of the automaton (instead of the set $F$). A word $w \in \Sigma^\omega$ is accepted iff there is a run $p$ with $\Inf(p) \in \T$, where $\Inf(p)$ is the set of infinitely often reached states of the run $p$.

We write:
\[ \A = (Q, \Sigma, E, i, \T) . \]

\subsection{Rabin automaton}
A Rabin automaton is a tuple $\A = (Q, \Sigma, E, i, \R)$, where $(Q,\Sigma,E)$ is a deterministic automaton, $i$ is the initial state and $\R = \{(L_j, U_j) | j \in J\}$ is a family of pairs of state-sets. A run $p$ is successfull iff it starts in $i$ and there is an index $j \ in J$ such that $p$ reaches $U_j$ infinitely often and $L_j$ only finitely often. If the automaton is finite, this is equivalent to
\[ \Inf(p) \cap L_j = \emptyset \ \text{and} \ \Inf(p) \cap U_j \neq \emptyset . \]

\subsection{Staiger Wagner class of $\K$}

\section{Operations: $*$-language $K$ to $\omega$-language $L_\omega (K)$}
\subsection{...}
a)
* alle Sprachen $K \dot \Sigma^\omega = \ext(K)$, $K \in \K$

* offene G

* Staiger Wagner Klasse
http://de.wikipedia.org/wiki/Staiger-Wagner-Automat
Erich Grädel, Wolfgang Thomas und Thomas Wilke (Herausgeber), Automata, Logics, and Infinite Games, LNCS 2500, 2002, Seite 20 (auf englisch)
http://www.automata.rwth-aachen.de/material/skripte/areas-english.pdf - s.53

a')
dual $\overline{K}$ = $\omega$-Wörter, deren alle Präfixe in $K$ sind

b) Sprachen $\lim \K$
BC Muller-erkennbare
(BC: boolean closure ?)

b') von einer Stelle an alle Prefixe in $K$

c) Kleene-Closure

alle der Form $\cup_{i=1}^n U_i \dot V_i^\omega$, $U_i, V_i \in \K$

d) $\K$ nicht suffix sensitiv

$K \in \K \Rightarrow K \dot \Sigma^* \in \K$  
