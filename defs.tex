% Quelle: Automata, Semigroups, Logic and Games - Pin, Perrin

\section{Automat}

Ein \defword{Automat} $\A$ auf dem Alphabet $\Sigma$ ist gegeben durch eine Menge $Q$ von Zuständen und einer Teilmenge $E \subset Q \times A \times Q$ von Transitionen. Außerdem ist in der Regel eine Teilmenge $I \subset Q$ von Startzuständen und eine Teilmenge $F \subset Q$ von Endzuständen gegeben.

Wir schreiben dafür: $\A = (Q, \Sigma, E, I, F)$.

Der Automat ist endlich genau dann, wenn $Q$ und $\Sigma$ endlich sind.

Der Automat ist deterministisch, wenn $E$ eine Menge von Funktionen $Q \times A \rightarrow \Q$ und wenn $|I| = 1$ sind.

\subsection{Pfad}
Zwei Transitionen $(p,a,q), (p',a',q') \in E$ sind aufeinanderfolgend, wenn $q=p'$.

Ein Pfad in dem Automat $\A$ ist eine Folge von aufeinanderfolgenden Transitionen, geschrieben als:
$q_0 \rightarrow^{a_0} q_1 \rightarrow^{a_1} q_2 \dots$

\subsection{Akzeptanz von endlichen Wörtern}

Ein Automat $\A = (Q, \Sigma, E, I, F)$ \defword{akzeptiert} ein endliches Wort $w = (a_0,a_1,...,a_n) \in \Sigma^*$ genau dann, wenn es einen Pfad $q_0 \rightarrow^{a_0} q_1 \rightarrow^{a_1} q_2 \dots \rightarrow^{a_n} q_{n+1}$ gibt mit $q_0 \in I$ und $q_{n+1} \in F$.

Die Sprache $L^*(\A)$ ist definiert als die Menge aller Wörter, die von $\A$ akzeptiert werden.

\section{$*$-Sprachklassen}
Die $*$-Sprachklasse ist die Menge aller Sprachen von Wörtern $w \in \Sigma^*$, also die Menge von Sprachen von endlichen Wörtern.

\subsection{reguläre Sprachen}
Eine Sprache ist genau dann regulär, wenn sie von einem endlichen Automat erkannt wird.

\subsection{piece-wise testable}


\subsection{$k$-locally testable}


\subsection{dot-depth-$n$}
\subsection{starfree}
\subsection{locally modulo testable}
\subsection{$R$-trivial}
\subsection{endlich / co-endlich}
\subsection{endwise testable}

\section{$\omega$-Sprachklassen}
\subsection{Büchi Automat}
Ein Automat $\A = (Q, \Sigma, E, I, F)$ \defword{Büchi-akzeptiert} ein Wort $w = (a_0,a_1,a_2,...) \in \Sigma^\omega$ genau dann, wenn es einen unendlichen Pfad $q_0 \rightarrow^{a_0} q_1 \rightarrow^{a_1} q_2 \rightarrow^{a_2} q_3 ...$ gibt mit $q_0 \in I$ und $\{ q_i | q_i \in F \}$ unendlich, also der unendlich oft einen Zustand $F$ erreicht.

Die Sprache $L^\omega(\A)$ ist definiert als die Menge aller unendlichen Wörter, die von $\A$ Büchi-akzeptiert werden.

Man bezeichnet einen Automaten $\A$ als Büchi Automat, wenn man von der Büchi-Akzeptanz ausgeht.

\subsection{Muller Automat}
Ein Muller Automat $\A$ ist ein endlicher, deterministischer Automat mit Muller Akzeptanzbedingung und einer Menge $\T \in 2^Q$, genannt die Tabelle des Automaten (anstatt der Menge $F$). Dabei wird ein Wort $w \in \Sigma^\omega$ akzeptiert genau dann, wenn es einen entsprechenden Pfad $p$ gibt mit $\Inf(p) \in \T$, wobei $\Inf(p)$ die Menge der unendlich oft besuchten Zustände ist.

Wir schreiben $\A = (Q, \Sigma, E, i, \T)$.

\subsection{Rabin Automat}
Ein Rabin Automat ist ein Tuple $\A = (Q, \Sigma, E, i, \R)$, wobei $(Q,\Sigma,E)$ ein deterministischer Automat ist, $i$ ist der Startzustand und $\R = \{(L_j, U_j) | j \in J\}$ ist eine Familie von Paren von Zustandsmengen. Ein Pfad $p$ ist erfolgreich, wenn er in $i$ beginnt und wenn es einen Index $j \ in J$ gibt, so dass $p$ unendlich oft $U_j$ besucht und nur endlich oft $L_j$. Ist der Automat endlich, so ist dies äquivalent mit
\[ \Inf(p) \cap L_j = \emptyset \ \text{und} \ \Inf(p) \cap U_j \neq \emptyset . \]

\subsection{Staiger Wagner Klasse zu $\K$}

\section{Operationen: von $*$-Sprache $K$ zu $\omega$-Sprache $L_\omega (K)$}
\subsection{...}
a)
* alle Sprachen $K \dot \Sigma^\omega = \ext(K)$, $K \in \K$

* offene G

* Staiger Wagner Klasse
http://de.wikipedia.org/wiki/Staiger-Wagner-Automat
Erich Grädel, Wolfgang Thomas und Thomas Wilke (Herausgeber), Automata, Logics, and Infinite Games, LNCS 2500, 2002, Seite 20 (auf englisch)
http://www.automata.rwth-aachen.de/material/skripte/areas-english.pdf - s.53

a')
dual $\overline{K}$ = $\omega$-Wörter, deren alle Präfixe in $K$ sind

b) Sprachen $\lim \K$
BC Muller-erkennbare
(BC: boolean closure ?)

b') von einer Stelle an alle Prefixe in $K$

c) Kleene-Closure

alle der Form $\cup_{i=1}^n U_i \dot V_i^\omega$, $U_i, V_i \in \K$

d) $\K$ nicht suffix sensitiv

$K \in \K \Rightarrow K \dot \Sigma^* \in \K$  
