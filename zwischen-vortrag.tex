%\documentclass[a4paper,10pt]{article}
\documentclass{beamer}

% http://www2.informatik.hu-berlin.de/~mischulz/beamer.html

\usepackage[ngerman]{babel}
\usepackage[utf8x]{inputenc}
\usepackage{amsmath,amsfonts,amssymb}
\usepackage{tikz}

% for PDF
\subject{a structure theory for Omega-languages}
\keywords{Omega, language}

\title[Kurzform]{a structure theory for $\omega$-languages}
%\subtitle[Kurzform]{Untertitel}
\author[A. Zeyer]{Albert Zeyer}
\institute[i7 RWTH]{Lehrstuhl für Informatik 7}
\date{\today}
%\logo{\pgfimage[width=2cm,height=2cm]{hulogo}}
%\titlegraphic{\includegraphics[width=2cm,height=2cm]{hulogo}}

 
\begin{document}

\frame{\titlepage}

%\frame{
%	\frametitle{Inhaltsverzeichnis}
%	\tableofcontents
%	[pausesections]
%}

\begin{frame}[<+->]{Introduction}
\begin{itemize}
\item consider regular $*$-languages
\item look at $* \rightarrow \omega$ language operators like lim, ext, BC of them, etc.
\item compare the resulting $\omega$-languages
\end{itemize}
\end{frame}

\begin{frame}{Diagram}
\begin{tikzpicture}
\pgftransformscale{.55}

% http://www.texample.net/tikz/examples/complexity-classes/

%%% HELP LINES - uncomment to design/extend
% \draw[step=1cm,gray,very thin] (-10,0) grid (10,12);
% \node at (0,0) {\textbf{(0,0)}};

%% Horizontal bar
\draw[very thick] (10,0) -- (-10,0);

% BC lim
\draw (-9.5,0) parabola bend (0,11) (9.5,0);
\node at (0,9) {BC lim};

% lim
\draw (-9,0) parabola bend (-3,8) (9,0);
\node[rotate=20] at (-3,7) {lim};

% dual-lim
\draw (-9,0) parabola bend (3,8) (9,0);
\node[rotate=-20] at (3,7) {dual-lim};

% BC ext
\draw (-6.5,0) parabola bend (0,6) (6.5,0);
\node at (0,5) {BC ext};

% ext
\draw (-6,0) parabola bend (-2,4) (6,0);
\node[rotate=20] at (-2,3.5) {ext};

% dual-ext
\draw (-6,0) parabola bend (2,4) (6,0);
\node[rotate=-20] at (2,3.5) {dual-ext};

\end{tikzpicture}

\end{frame}

\begin{frame}
\begin{itemize}
\item instead of regular $*$-languages, look at other $*$-language classes
\item does it result in the same relations as in the diagram? are the enclosures strict?
\end{itemize}
\end{frame}

\begin{frame}
\begin{itemize}
\item instead of regular $*$-languages, look at other $*$-language classes
\item does it result in the same relations as in the diagram? are the enclosures strict?
\end{itemize}
\end{frame}

\end{document}
