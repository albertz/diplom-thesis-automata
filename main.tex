

Hauptfrage: Für welche $\K$ ergibt sich eine andere Sprache als bei $\K = \Reg$.

\section{$*$-Sprachklassen}
\subsection{regular}
\subsection{piece-wise testable}
\subsection{$k$-locally testable}
\subsection{dot-depth-$n$}
\subsection{starfree}
\subsection{locally modulo testable}
\subsection{$R$-trivial}
\subsection{endlich / co-endlich}
\subsection{endwise testable}

\section{$\omega$-Sprachklassen}
\subsection{Staiger Wagner Klasse zu $\K$}

\section{Operationen: von $*$-Sprache $K$ zu $\omega$-Sprache $L_\omega (K)$}
\subsection{...}
a)
* alle Sprachen $K \dot \Sigma^\omega = \ext(K)$, $K \in \K$

* offene G

* Staiger Wagner Klasse
http://de.wikipedia.org/wiki/Staiger-Wagner-Automat
Erich Grädel, Wolfgang Thomas und Thomas Wilke (Herausgeber), Automata, Logics, and Infinite Games, LNCS 2500, 2002, Seite 20 (auf englisch)
http://www.automata.rwth-aachen.de/material/skripte/areas-english.pdf - s.53

a')
dual $\overline{K}$ = $\omega$-Wörter, deren alle Präfixe in $K$ sind

b) Sprachen $\lim \K$
BC Muller-erkennbare
(BC: boolean closure ?)

b') von einer Stelle an alle Prefixe in $K$

c) Kleene-Closure

alle der Form $\union_{i=1}^n U_i \dot V_i^\omega$, $U_i, V_i \in \K$

d) $\K$ nicht suffix sensitiv

$K \in \K \Rightarrow K \dot \Sigma^* \in \K$  
