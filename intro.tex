%!TEX root =  index.tex

\section{Introduction}

The study of formal languages and finite-state automata theory is very old and fundamental in theoretical computer science. Regular expressions were introduced by Kleene in 1956 (\cite{Kleene56}). Research on the connection between formal languages, automata theory and mathematical logic began in the early 1960's by Büchi (\cite{Buchi60}). Good introductions into the theory are \cite{FinAutLogR109} and \cite{LangAutLogicR102}.

We call languages over finite words the $*$-languages. Likewise, $\omega$-languages are over infinite words.

The class of regular $*$-languages is probably the most well studied language class. Its expressiveness is exactly equivalent to the class of finite-state automata as well as regular expressions. For many applications, less powerful subsets of the regular $*$-languages are interesting, like starfree $*$-languages, locally testable $*$-languages, etc., as well as more powerful supersets, like context-free $*$-languages.

The research on $\omega$-languages and their connection to finite-state automata began a bit later by Büchi \cite{DecisionSOR111} and \cite{Muller63}. As for the $*$-languages, the most well studied $\omega$-language class are the regular $\omega$-languages. Good introductions into these theories are \cite{AutInfObjsR103}, \cite{InfCompR101}, \cite{OmLangR108} and \cite{InfWordsR110}.

The acceptance-condition in automata for $*$-languages is straight-forward. If we look at $\omega$-languages, several different types of automata and their acceptance have been thought of, like Büchi-acceptance or Muller-acceptance, or E-acceptance and A-acceptance.

For all types, we can also argue with equivalent language-theoretical operators which operate on a $*$-language and transform them into an $\omega$-language. We will study the equivalences in more detail. The most important operators are $\ext$, $\lim$ and boolean combinations of those.

Depending on the $* \rightarrow \omega$ language operator or the $\omega$-automaton acceptance condition, we get different $\omega$-language classes. This was studied earlier already in detail for the class of regular $*$-languages. E.g., we get the result $\BC \ext \Langreg \subsetneqq \BC \lim \Langreg$ and $\lim \cap \dlim \Langreg = \BC \ext \Langreg$. In terms of $\omega$-automata, that is that boolean combinations of E-automata are strictly less powerful than boolean combinations of deterministic Büchi automata. Those in turn are equivalent to Muller automata.

When we look at other $*$-language classes, like piecewise testable $*$-languages and the different ways to transform them into $\omega$-languages, we can get different results. E.g., $\BC \ext \Lang(\PT) = \BC \lim \Lang(\PT)$. This study is the main topic of this thesis. I.e. we try to derivate some generic conditions on a $*$-language class $\Lang$ under which we get similar properties to the regular $*$-languages. And we will see many examples where the $\omega$-language classes have different relations to each other.

In \cref{chapter:regOmegaLangs}, we introduce the basic terminology of automata theory and language theory. The class of regular $*$-languages is introduced. Then, we go forward to the introduction of $\omega$-languages and we define and characterize the class of regular $\omega$-languages. We also introduce all $* \rightarrow \omega$ language operators used in this thesis. The chapter ends with a classification of regular $\omega$-languages into $\ext \Langreg$, $\BC \ext \Langreg$, $\lim \Langreg$ and $\BC \lim \Langreg$. We see that we have strict inclusions on all those $\omega$-language classes. The diagram in \cref{regomega-diagram} visualizes these relations.

In \cref{general-results}, we derivate generic conditions for arbitrary $*$-language classes $\Lang$ under which we get the same inclusions or even strict inclusions as in the $\Langreg$ case. This is the main foundation of this thesis. The chapter starts with some generic lemmas, then introduces some closure properties on $\Lang$ which are necessary conditions for many of the theorems. The chapter ends in \cref{gen:R} with the study of a more specific case of $*$-language classes: Languages defined as finite union of equivalence classes of some congruence relation $R \subseteq \Sigma^*\times\Sigma^*$ on words.

\Cref{concrete-results} studies concrete well-known $*$-language classes. We mostly concentrate on subsets of regular languages. We will both study the relations and properties in concrete as well as apply the results from \cref{general-results}. Some of the language classes are defined via congruence relations, e.g. locally testable or piecewise testable languages, so we can apply the results from \cref{gen:R}.

\Cref{chapter:conclusion} finishs with a conclusion.
