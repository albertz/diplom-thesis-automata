%!TEX root =  index.tex

\section{Introduction}

The study of formal languages and finite-state automata theory is very old and fundamental in theoretical computer science. Regular expressions were introduced by Kleene in 1956 (\cite{Kleene56}). Research on the connection between formal languages, automata theory and mathematical logic began in the early 1960's by Büchi (\cite{Buchi60}). Good introductions into the theory are \cite{FinAutLogR109} and \cite{LangAutLogicR102}.

We call languages over finite words the $*$-languages. Likewise, $\omega$-languages are over infinite words. Words are just sequences of input symbols.

The class of regular $*$-languages is probably the most well studied language class. Its expressiveness is exactly equivalent to the class of finite-state automata as well as regular expressions. The connection between finite-state automata and regular languages was established by S. C. Kleene in \cite{Kleene56}. A more generic concept of finite-state automata, including the non-deterministic case, was introduced by M. O. Rabin and D. Scott in \cite{FinAutRabin59}. It was inspired as a strict subset of the Turing machine which has infinite many states given by its memory. Such finite-state automata get a sequence of input symbols and change their state on each input symbol. Depending on their state, we say that such automaton accepts a given word. Thus, an automaton is representing a language. And the class of languages given by such finite-state automata is equal to the class of regular languages.

For many applications, less powerful subsets of the regular $*$-languages are interesting, like starfree $*$-languages, locally testable $*$-languages, etc., as well as more powerful supersets, like context-free $*$-languages.

The research on $\omega$-languages and their connection to finite-state automata began a bit later by Büchi \cite{DecisionSOR111} and \cite{Muller63}. As for the $*$-languages, the most well studied $\omega$-language class are the regular $\omega$-languages. Good introductions into these theories are \cite{AutInfObjsR103}, \cite{InfCompR101}, \cite{OmLangR108} and \cite{InfWordsR110}. In contrast to the finite-word acceptance of automata, one can think of several different infinite-word acceptance conditions which lead to different automata, most importantly the Büchi and Muller automata. A central result by Robert McNaughton in \cite{McNaughton66} is the equivalence of non-deterministic Büchi automata and deterministic Muller automata.

Landweber established a strict hierarchy in \cite{Landweber69} on subsets of the regular $\omega$-languages, given by other/simpler infinite-word acceptance conditions in finite-state automata, namely the E-/A-acceptance condition and deterministic Büchi and co-Büchi automata.

For all types, we can also argue with equivalent language-theoretical operators which operate on a $*$-language and transform them into an $\omega$-language. We will study the equivalences in more detail. The most important operators are $\ext$, $\lim$, in some way equivalent to E-acceptance and deterministic Büchi acceptance, and boolean combinations of those. For some $*$-language $L$, $\ext L$ are all infinite words where some prefix is in $L$ and $\lim L$ are all finite words where infinitely many prefixes are in $L$.

Depending on the $* \rightarrow \omega$ language operator or the $\omega$-automaton acceptance condition, we get different $\omega$-language classes. This was studied earlier already in detail for the class of regular $*$-languages. E.g., we get the result that the class of boolean combinations of $\ext$-languages is a strict subset of the class of boolean combinations $\lim$-languages. In terms of $\omega$-automata, that is that boolean combinations of E-automata are strictly less powerful than boolean combinations of deterministic Büchi automata. Those in turn are equivalent to Muller automata. This is basically Landweber's Theorem from \cite{Landweber69}. We can also see that the class of boolean combinations of $\ext$-languages, which can be represented by boolean combinations of E-automata is equivalent to the class of languages which can be recognized by both deterministic Büchi and deterministic co-Büchi automata. This is the result from Staiger and Wagner in \cite{StaigerW74}.

When we look at other $*$-language classes, like piecewise testable $*$-languages and the different ways to transform them into $\omega$-languages, we can get different results. E.g., $\BC \ext \Lang(\PT) = \BC \lim \Lang(\PT)$. This study is the main topic of this thesis. I.e. we try to derive some generic conditions on a $*$-language class $\Lang$ under which we get similar statements to the regular $*$-languages. And we will see many examples where the $\omega$-language classes have different relations to each other than in the regular case.

In \cref{chapter:regOmegaLangs}, we introduce the basic terminology of automata theory and language theory. The class of regular $*$-languages is introduced. Then, we go forward to the introduction of $\omega$-languages and we define and characterize the class of regular $\omega$-languages. We also introduce all $* \rightarrow \omega$ language operators used in this thesis. The chapter ends with a classification of regular $\omega$-language classes into $\ext \Langreg$, $\BC \ext \Langreg$, $\lim \Langreg$ and $\BC \lim \Langreg$. We see that we have strict inclusions on these $\omega$-language classes. The diagram in \cref{regomega-diagram} visualizes these relations.

In \cref{general-results}, we derive generic conditions for arbitrary $*$-language classes $\Lang$ under which we get the same inclusions or even strict inclusions as in the $\Langreg$ case. This is the main foundation of this thesis. The chapter starts with some generic lemmas, then introduces some closure properties on $\Lang$ which are necessary conditions for many of the theorems. The chapter ends in \cref{gen:R} with the study of a more specific case of $*$-language classes: Languages defined as finite unions of equivalence classes of some congruence relation $R \subseteq \Sigma^*\times\Sigma^*$ on words.

\Cref{concrete-results} studies concrete well-known $*$-language classes. We mostly concentrate on subsets of the class of regular languages. We will both study the relations and properties in concrete as well as apply the results from \cref{general-results}. Some of the language classes are defined via congruence relations, e.g. locally testable or piecewise testable languages, so we can apply the results from \cref{gen:R}.

\Cref{chapter:conclusion} finishes with a conclusion.
