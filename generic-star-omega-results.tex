\section{General results}
\label{general-results}

%S401: TODO

Let $\Lang$ be a $*$-language class. We start with some very basic results on language operators.

\subsection{Background}
\begin{lemma}
\label{gen:general}
Let $L,A,B \in \Lang$.
\begin{enumerate}
\item $\ext L = L \cdot \Sigma^\omega$
\item $\ext L = \lim(L \cdot \Sigma^*)$ %S303
\item $\ext L = \dlim(L \cdot \Sigma^*)$ %S303, S305.4
\item $- \lim(-L) = \dlim(L)$
\item $\dlim L \subseteq \lim L$
\item $\lim A \cup \lim B = \lim(A \cup B)$ %S305.4
\item $\dlim A \cup \dlim B \subseteq \dlim(A \cup B)$ \newline %S305.4
There is no equality in general: $A = (00)^*$, $B = (00)^*0$.
\end{enumerate}
\begin{proof}
\begin{enumerate}
\item[1.-5.] They all follow directly from the definition.
\item[6.] %S305.4
\begin{align*}
& \alpha \in \lim A \cup \lim B \\
\Leftrightarrow \ & \existsinf n \colon \alpha[0,n] \in A \ \vee \ \existsinf n \colon \alpha[0,n] \in B \\
\Leftrightarrow \ & \existsinf n \colon \alpha[0,n] \in A \cup B \\
\Leftrightarrow \ & \alpha \in \lim A \cup B
\end{align*}
\item[7.] %S305.4
\begin{align*}
& \alpha \in \dlim A \cup \dlim B \\
\Leftrightarrow \ & \exists N \colon \forall n \ge N \colon \alpha[0,n] \in A \ \vee \ 
\exists N \colon \forall n \ge N \colon \alpha[0,n] \in B \\
\Rightarrow \ & \exists N \colon \forall n \ge N \colon \alpha[0,n] \in A \cup B
\end{align*}
\end{enumerate}
\end{proof}
\end{lemma}

For $\omega$ automata, we already know that non-determinism can be more powerful than determinism (see \ref{reg-omega-lang}): The class of non-deterministic Büchi automata can accept clearly more languages than the class of determinstic Büchi automata. E.g. $L^\omega := (a+b)^* b^\omega \in \LangOreg$ cannot be recognised by deterministic Büchi automata, i.e. $L^\omega \not\in \lim \Langreg$.

Luckily, for E- and A-acceptance, this is not the case as we see below. This matches the intuition that E/A-acceptance doesn't really tell something about infinite properties of words but Büchi/Muller does. And when talking about finite words, we already know that non-deterministic and deterministic automata are equaly powerful (see chapter \ref{intro:reglang}).

%307.12
\begin{lemma}
\label{gen:e-determinism}
The $\omega$-language-class accepted by deterministic E-automata is equal to non-deterministic E-automata. I.e., for every non-deterministic E-automaton, we can construct an equivalent deterministic E-automaton. The same goes for A-automata.
\begin{proof} Let $\A^N$ be any non-deterministic automaton and $\A^D$ an ($*$-)equivalent deterministic automaton. Then:
\begin{align*}
& \alpha \in L^{\omega}_E(\A^N) \\
\Leftrightarrow \ & \exists n \colon \alpha[0,n] \in L(\A^N) \\
\Leftrightarrow \ & \exists n \colon \alpha[0,n] \in L(\A^D) \\
\Leftrightarrow \ & \alpha \in L^{\omega}_E(\A^D)
\end{align*}
$\A^N$ can be interpreted as an arbitary E-automata and we have shown that we get an equivalent deterministic E-automata.

For A-automata, the proof is analogue.
\end{proof}
\end{lemma}

\

%S302.1:
We are interested in relations like $\BC \ext \Lang \overset{?}{\subsetneqq} \BC \lim \Lang$ or $\ext \Lang \overset{?}{\subsetneqq} \lim \Lang$. With $\Lang = \Set{\Set{a}}$, we realize that even $\ext \Lang \subseteq \lim \Lang$ is not true in general ($\ext \Set{\Set{a}} = \Set{a \Sigma^\omega} \neq \emptyset = \lim \Set{\Set{a}}$). In \ref{gen:non-suffix-sens}, we see a sufficient condition for this property, though.

\

We will formulate some properties of interest in a general form for a $*$-language class $\Lang$ which all hold for $\Langreg$. We get some general results based on these properties in chapter \ref{general-results}.

%S303.E3, S218, R101
Let $L,A,B \in \Lang$.
\begin{itemize}
\item E1: closure under suffix-insensitiveness: $L \cdot \Sigma^* \in \Lang$
\item E2a: closure under union: $A \cup B \in \Lang$
\item E2b: closure under intersection: $A \cap B \in \Lang$
\item E3: closure under complementation/negation: $- L \in \Lang$

\item
%S303.1:
In some proofs, e.g. in \ref{gen:staiger-wagner} or \ref{gen:kleene-star}, we have an automaton based on some language of the language class and we do some modifications on it, e.g. we modify the final state set. If we stay in the language class, we call this the E4 property. Formally:

E4: $\forall \text{ deterministic automaton } \A = (Q,q_0,\Delta,F), L^*(\A) = L \colon \forall F' \subseteq Q: L^*((Q,q_0,\Delta,F')) \in \Lang$

For $\Langreg$, this property holds obviously.

%S303.1.a:
For $\Lang^*(\mathrm{FO[<]})$, it does not hold:

  \begin{tikzpicture}[%
    >=stealth,
	shorten >=1pt,
	node distance=2cm,
%    on grid,
    auto,
%    state/.append style={minimum size=2em},
%    thick
  ]
    \node[state] (S)              {};
    \node[state,double] (A) [right of=S] {$A$};
    \node[state] (B) [below of=S] {$B$};
    \node[state] (C) [right of=B] {};

    \path[->] (S) +(-1,0) edge (S)
              (S)         edge              node {$a$} (A)
              (S)         edge              node {$b$} (B)
              (B)         edge  [bend left]            node {$b$} (C)
              (C)         edge  [bend left]            node {$b$} (B)
              ;
  \end{tikzpicture}
  
This is a deterministic automaton for the language $\Set{a} \in \Lang^*(\mathrm{FO[<]})$. If you make $B$ also a final state, we get the language $a + b(bb)^* \not\in \Lang^*(\mathrm{FO[<]})$.

So, E4 seems too restricted.

%S303.1: TODO: weniger ...

\end{itemize}


\subsection{Classification for arbitrary language classes}

%\subsection{non suffix sensitive}
% S303, S101a:
\begin{lemma}
\label{gen:non-suffix-sens}
If $\Lang$ is closed under suffix-insensitiveness,
\[ \ext \Lang \subseteq \lim \Lang \cup \dlim \Lang . \]
\begin{proof}
For $L \in \Lang$, we have $\ext L = \lim L \Sigma^* = \dlim L \Sigma^*$.
\end{proof}
\end{lemma}

%\subsection{$\BC \ext \subseteq \BC \lim$}
%S302:
\begin{lemma}
\label{gen:S302a} If we have $\ext \Lang \subseteq \lim \Lang$, then we also have
\[ \BC \ext \Lang \subseteq \BC \lim \Lang . \]
\begin{proof}
From $\ext \Lang \subseteq \lim \Lang$, it directly follows $\Set{-\ext L}{L \in \Lang} \subseteq \Set{-\lim L}{L \in \Lang}$. Thus, it also follows the claimed inequality.
\end{proof}
\end{lemma}

%\subsection{$\dext \subset \dlim$}
%S302:
\begin{lemma}
If we have $\ext \Lang \subseteq \lim \Lang$ and let $\Lang$ be closed under negation. Then we have
\[ \dext \Lang \subseteq \dlim \Lang . \]
\begin{proof}
Let $L \in \Lang$. Then $ \dext L = - \ext (-L) $. Because of the negation closure, we also have $-L \in \Lang$ and $\ext(-L) \in \ext \Lang$. Thus $\lim(-L) \in \lim \Lang$. Thus, $-\lim(-L) \in \Set{-\lim A}{A \in \Lang} = \Set{\dlim A}{-A \in \Lang}$. Because of the negation closure, this is equal to $\Set{\dlim A}{A \in \Lang} = \dlim \Lang$. I.e. $\dlim L \in \dlim \Lang$.
\end{proof}
\end{lemma}

Note that we needed the negation closure in the proof. This is in contrast to \ref{gen:S302a}, where it directly follows. We have to be careful about the difference $- \ext \Lang := \Set{-\ext L}{L \in \Lang} \neq \dext \Lang$ (in general, if $\Lang$ is not closed under negation).

%\subsection{union, intersection}
%S303:
\begin{lemma}
\begin{itemize}
\item
$\Lang$ closed under union $\Rightarrow \bigcup \ext \Lang \subseteq \ext \Lang$.
\item
$\Lang$ closed under intersection $\Rightarrow \bigcap \ext \Lang \subseteq \ext \Lang$.
\end{itemize}
\begin{proof}
Let $A,B \in \Lang$. Then we have $\ext(A\cup B) = \ext(A) \cup \ext(B)$ and $\ext(A\cap B) = \ext(A) \cap \ext(B)$.
\end{proof}
\end{lemma}

%\subsection{$op \cup \overline{op} \subsetneqq \BC op$}
\begin{lemma}
If there is $L_\Sigma \in \Lang_\Sigma$ with $L_\Sigma \in \ext \Lang_\Sigma, L_\Sigma \not\in \dext \Lang_\Sigma$ and E3 (closed under negation) holds for $\Lang$:
\begin{itemize}
\item[$\Rightarrow$] $-L_\Sigma \in \dext \Lang_\Sigma, -L_\Sigma \not\in \ext \Lang_\Sigma$
\item[$\Rightarrow$] $L_{\Sigma_1} \cup -L_{\Sigma_2} \in \BC \ext \Lang_{\Sigma_1 \dot{\cup} \Sigma_2}$ \newline
$L_{\Sigma_1} \cup -L_{\Sigma_2} \not\in \ext \cup \dext \Lang_{\Sigma_1 \dot{\cup} \Sigma_2}$
\end{itemize}
Thus,
\[ \ext \cup \dext \Lang \subsetneqq \BC \ext \Lang . \]
\end{lemma}

\begin{lemma}
Similarily, if there is $L_\Sigma \in \lim \Lang_\Sigma, L_\Sigma \not\in \dlim \Lang_\Sigma$ and E3 holds for $\Lang$:
\begin{itemize}
\item[$\Rightarrow$] $-L_\Sigma \in \dlim \Lang_\Sigma, -L_\Sigma \not\in \lim \Lang_\Sigma$
\item[$\Rightarrow$] $L_{\Sigma_1} \cup -L_{\Sigma_2} \in \BC \lim \Lang_{\Sigma_1 \dot{\cup} \Sigma_2}$ \newline
$L_{\Sigma_1} \cup -L_{\Sigma_2} \not\in \lim \cup \dlim \Lang_{\Sigma_1 \dot{\cup} \Sigma_2}$
\end{itemize}
Thus,
\[ \lim \cup \dlim \Lang \subsetneqq \BC \lim \Lang . \]
\end{lemma}

%\subsection{$\BC \ext = \lim \cap \dlim$}
\begin{mydef}
A Staiger-Wagner automaton (weak Muller automaton) is of the same form $\A = (Q,\Sigma,q_0,\delta,\F)$ as a Muller automaton with the acceptance condition that a run $\rho$ is accepting if and only if $\Occ(\rho) := \Set{q \in Q \colon \text{q occurs in } p} \in \F$.  (R101,Def.61,p.43)
\end{mydef}

\begin{lemma}
%S405:
We see (R101,Th.63+64,p.44) that the class of Staigner-Wagner-recognized languages is exactly the class $\BC \ext \Langreg$ and also $\lim \cap \dlim \Langreg$.
\end{lemma}

We are now formulating a more general and direct proof for the $\BC \ext \Lang = \lim \cap \dlim \Lang$ equality without Staiger-Wagner-automata (where some of the ideas are loosely based on (R101,Th.63+64,p.44)).

\

\begin{lemma}
\label{gen:staiger-wagner}
\[ \BC \ext \Lang = \lim \cap \dlim \Lang \]

\begin{proof}
%S305:
First, we show $\lim \cap \dlim \Lang \subseteq \BC \ext \Lang$.

Let $\tilde L \in \lim \cap \dlim \Lang$, i.e. there are deterministic automaton $\A$ and $\overline \A$ so that $L^\omega_{\text{Büchi}}(\A) = L^\omega_{\text{co-Büchi}}(\overline \A) = \tilde L$. Let $Q,\overline Q$ be the states of $\A,\overline \A$. Now look at the product automaton $\A \times \overline \A =: \Ax$ with states $Q \times \overline Q$ and final states $F \times \overline F \subseteq Q \times \overline Q$. $\Ax$ is also deterministic.

%Let $(q,\overline q) \in Q \times \overline F$. If $\overline q$ is not part of a loop in $\A$, we can just take it away from $\overline F$ and get the same $\omega$-language (no matter if Büchi or co-Büchi). Thus assume that all $\overline q \in \overline F$ are part of a loop.

%Let $\alpha \in \tilde L$. Then $\rho(\alpha)$ has infinity many occurences in $F$ and there is some $N$ so that $\overline \rho(\alpha)[N,\infty)$ is only in $\overline F$. Let $\beta \not\in \tilde L$. Then there is some $N$ so that $\rho(\alpha)[N,\infty)$ is only in $Q-F$.

%S305.1
In $\Ax$, we have
\begin{align*}
& \alpha \in \tilde L \\
\Leftrightarrow \ & \forall N \colon \exists n \ge N \colon \overx{\rho}(\alpha)[n] \in F \times \overline Q \\
\Leftrightarrow \ & \exists N \colon \forall n \ge N \colon \overx{\rho}(\alpha)[n] \in Q \times \overline F
\end{align*}

Look at strongly connected component (SCC) $S$ in $\Ax$. We have $S \cap F \times \overline Q \neq \emptyset$, iff $S$ accepts. It follows that all states in $S$ are finite states in $\overline \A$, i.e. $S \cap Q \times \overline F = S$.

Single $\overx{q} \in \overx{Q}$ which are not part of a SCC can be ignored. For the acceptance of infinte words, only SCCs are relevant. For $S$, define $S_+ := \Set{\overx{q} \in \overx{Q}-S}{\overx{q} \text{ can be visited after } S}$.

Then we have
\begin{align*}
\tilde L = \bigcup_{\text{SCC } S} & S \text{ will be visited} \ \wedge \\
& \text{all states of } S \text{ will be visited forever after some step} \ \wedge \\
& S_+ \text{ will not be visited} .
\end{align*}

%S305.2:
$S$ will be visited: Let $S$ exactly be the finite states. This interpreted as an E-automaton $\A^E_S$ is exactly the condition.

Only the allowed states will be visited but nothing followed after $S$: Mark $S$ and all states on all paths to $S$ as finite states. This as an A-automaton $\A^A_S$ is exactly the condition.

A similar negated condition might be simpler: Let $S_+$ be exactly the finite states. Interpret this as an E-automaton $\A^E_{S_+}$.

Then we have
\begin{align*}
\tilde L & = \bigcup_{\text{SCC } S} L^\omega_E (\A^E_S) \cap L^\omega_A (\A^A_S) \\
& = \bigcup_{\text{SCC } S} L^\omega_E (\A^E_S) \cap -L^\omega_E (\A^E_{S_+}) .
\end{align*}

Thus,
\[ \tilde L \in \BC \ext \Langreg . \]

%TODO...
Open question at this point: Is $L^*(\A^E_S), L^*(\A^E_{S_+}) \in \Lang$? With E4, this is obviously the case. However, E4 is too strong.

\

%S305.3:
Now let us show $\BC \ext \Lang \subseteq \lim \Lang$.

With E1, we get $\ext \Lang \subseteq \lim \Lang$ and $\ext \Lang \subseteq \dlim \Lang$. I.e. $\ext \Lang \subseteq \lim \cap \dlim \Lang$. Let us show that $\lim \cap \dlim \Lang$ is closed under BC.

Let $\tilde L_a,L_b \in \lim \cap \dlim \Lang$, i.e. $\exists L_{a1}, L_{a2}, L_{b1}, L_{b2} \in \Lang \colon \tilde L_a = \lim L_{a1} = \dlim L_{a2}, \tilde L_b = \lim L_{b1} = \dlim L_{b2}$. Let us show 1. $-\tilde L_a \in \lim \cap \dlim \Lang$, 2. $\tilde L_a \cup \tilde L_b \in \lim \cap \dlim \Lang$.
\begin{enumerate}
\item $-\tilde L_a = -\lim L_{a1} = \dlim -L_{a1}$, $-\tilde L_b = -\dlim L_{a2} = \lim -L_{a2}$. With E3 (closed under negation), we get
\[ -\tilde L_a \in \lim \cap \dlim \Lang . \]

\item
$\tilde L_a \cup \tilde L_b = \lim L_{a1} \cup \lim L_{b1} = \lim L_{a1} \cup L_{b1}$ (\ref{gen:general}). Thus, with E2a, we have
\[ \tilde L_a \cup \tilde L_b \in \lim \Lang . \]

The $\dlim\Lang$ case is harder.
%S305.5: language theoretical proof: TODO
%S305.6:
Let $\A_a$, $\A_b$ be deterministic automaton, so that $L^\omega_{\text{Büchi}}(\A_a) = L^\omega_{\text{co-Büchi}}(\A_a) = \tilde L_a$, $L^\omega_{\text{Büchi}}(\A_b) = L^\omega_{\text{co-Büchi}}(\A_b) = \tilde L_b$. Look at the product automaton $\A_a \times \A_b =: \Ax$. Then we have $L^\omega_{\text{Büchi}}(\Ax) = L^\omega_{\text{co-Büchi}}(\Ax) = \tilde L_a \cup \tilde L_b$.

Thus,
\[ \tilde L_a \cup \tilde L_b \in \dlim \Langreg . \]

Open question: Is $L^*(\Ax) \in \Lang$?

\end{enumerate}
\end{proof}
\end{lemma}

%S406.1: TODO...
%S402: TODO...

%\subsection{Kleene-star $= \BC \lim$}
\begin{lemma}
\label{gen:kleene-star}
\[ \BC \lim \Lang = \Kleene \Lang \]

\begin{proof}
%S407: TODO...
%S306, S306.1:
Let $U,V \in \Lang$. Look at the non-deterministic automaton $\A$ defined as:
\[
  \begin{tikzpicture}[%
    >=stealth,
	shorten >=1pt,
	node distance=2cm,
    auto,
  ]
    \node (U)              {$U$};
    \node (V) [right of=U] {$V \odot$};

    \path[->] (U) +(-1,0) edge (U)
              (U)         edge              node {$\epsilon$} (V)
              (V)         edge  [loop above]       node {} ()
              ;
  \end{tikzpicture}
\]

Then we have $L^\omega_{\text{Büchi}}(\A) = U \cdot V^\omega$.

Let us construct deterministic automata for $\A$ so that we can formulate 'V will be visited and not be left anymore' and 'finite states of the V-related automaton will be visited infinitely often' (or '$UV^*$ will be visited infinitely often').

In a constructed automaton, we must be able to tell wether we are in $U$ or we deterministically have been in $U$ the previous state. In a state power set construction, we can tell wether we are deterministically in $U$ or not. If we are non-determinstic and we may be in both $U$ or $V$ and we get an input symbol which determines that we have been in $U$, we might not be able to tell from the following power set. Example:

Let $U = (a+b)^*$, $V = \Set{b}$. I.e. $UV^\omega = \Set{\alpha \in \Set{a,b}^\omega}{\text{at one point in $\alpha$, there are only $b$s}}$. The non-deterministic automaton is:
\[
  \begin{tikzpicture}[%
    >=stealth,
	shorten >=1pt,
	node distance=2cm,
    auto,
  ]
    \node[state] (1)              {$1$};
    \node[state,double] (2) [right of=1] {$2$};
	
    \path[->]
    (1) +(-1,0) edge (1)
    (1) edge [loop above] node {$a,b$} ()
    (1) edge node {$\epsilon$} (2)
    (2) edge [loop above] node {$b$} ()
    ;
  \end{tikzpicture}
\]

Powerset construction: The initial state is $\Set{1,2}$. Then we have:
\begin{itemize}
\item $\Set{1,2} \overset{a}\rightarrow \Set{1,2}$
\item $\Set{1,2} \overset{b}\rightarrow \Set{1,2}$
\end{itemize}
This gives the $*$-language $\Set{a,b}^*$ and we cannot formulate $UV^\omega$ in any way from there.

In the construction, when we got the $a$ from $\Set{1,2}$, we knew that we have been deterministically in $1$, i.e. in $U$. We loose this information. To keep it, we introduce another state flag which exactly says wether we have determined that we have been in $U$. Thus, we construct an automaton with the states $\Power(Q) \times \B_{\text{det. been in $U$}}$, where $Q$ are the states from $\A$.

For the example, we get the initial state $(\Set{1,2},1)$. Then we have:
\begin{itemize}
\item $(\Set{1,2},1) \overset{a}\rightarrow (\Set{1,2},1)$
\item $(\Set{1,2},1) \overset{b}\rightarrow (\Set{1,2},0)$
\item $(\Set{1,2},0) \overset{a}\rightarrow (\Set{1,2},1)$
\item $(\Set{1,2},0) \overset{b}\rightarrow (\Set{1,2},0)$
\end{itemize}
This is the automaton
\[
  \begin{tikzpicture}[%
    >=stealth,
	shorten >=1pt,
	node distance=2cm,
    auto,
  ]
    \node[state] (1)              {$\Set{1,2},1$};
    \node[state] (2) [right of=1] {$\Set{1,2},0$};

    \path[->]
    (1) +(-1.5,0) edge (1)
    (1) edge [loop above] node {$a$} ()
    (1) edge [bend left] node {$b$} (2)
    (2) edge [bend left] node {$a$} (1)
    (2) edge [loop above] node {$b$} ()
    ;
  \end{tikzpicture}
\]

When we mark all states from $V$ and where we have not been deterministically in $U$ as final, this as a co-Büchi automaton gives exactly the condition 'V will be visited and not be left anymore'. Let $L_E$ be the $*$-language of this automata. Note that $L_E \neq UV^*$ in general and esp. in the example.

When we mark the final states as in the original non-deterministic automata, no matter about $\B_{\text{det. been in $U$}}$, with Büchi-acceptance, we get the condition '$UV^*$ will be visited infinitely often'. This is just $\lim UV^*$.

Together, we get $UV^\omega$, i.e.:
\[ \lim UV^* \cap \dlim L_E = UV^\omega \]

If we have $L_E \in \Lang$, it follows
\[ \Set{ \bigcup_{i=1}^n U_i \cdot V_i^\omega}{U_i, V_i \in \Lang} = \Kleene \Lang \subseteq \BC \lim \Lang . \]

Open question: Is $L_E \in \Lang$?

\

We also need to show $\BC \lim \Lang \subseteq \Kleene \Lang$.

%S306.3:
Show: $\lim \Lang \subseteq \Kleene \Lang$.

Proof: Let $\A$ be a deterministic Büchi automaton for some language $\tilde L = L^\omega_{\text{Büchi}}(\A) \in \Lang$ with final states $F$.

For all finite states $q \in F$: If $q$ is not part of a strongly connected component (SCC), we can ignore it. Let $S$ be the SCC where $q \in S$. Then the set of all $\alpha \in \Sigma^\omega$ which are infinitely often in $q$ can be described as $U_q \cdot V_q^\omega$, where $U_q$ is the set of words so that we arrive in $q$ and $V_q$ is the set of words so that we get from $q$ to $q$. Both sets are regular.

Thus,
\[ \tilde L = L^\omega_{\text{Büchi}}(\A) = \bigcup_{q \in F} U_q V_q^\omega . \]

Obviously, the Kleene-Closure is closed under union.

TODO: Show that Kleene-Closure is closed under negation. (S306.5) (Follows with non-det Büchi complementation but a more generic proof might be useful.)
\end{proof}
\end{lemma}

\subsection{Congruenced based language classes}

\subsubsection{Introduction}
\label{gen:R-automata}

\begin{mydef}
We define $\Lang(R)$ for an equivalence relation $R\subseteq\Sigma^* \times \Sigma^*$
\[ \Lang(R) := \Set{L \subseteq \Sigma^*}{L \text{ is finite union of $R$-equivalence-classes}} . \]
\end{mydef}

Examples of such language classes are locally testable (LT, section \ref{lang:LT}), locally threshold testable (LTT, section \ref{lang:LTT}) or piece-wise testable (PT, section \ref{lang:PT}) languages. At their definition, the word-relation basically tells wether a local test / piece-wise test can see a difference between two words.

%S307
If a language class $\Lang(R)$ is defined as finite union of equivalence classes of a relation $R \subseteq \Sigma^* \times \Sigma^*$ and
\begin{itemize}
\item the set of equivalent classes of $R$ is finite,
\item $R$ is a congruence relation, i.e. also $(v,w) \in R \ \Leftrightarrow \ (va,wa) \in R \ \ \forall a \in \Sigma$
\end{itemize}
then we can construct a canonical deterministic automaton $\A_R$ which has $S_R := \Sigma^* / R$ as states, $\left<\epsilon\right>_R$ is the initial state and the transitions are according to concatenation. Call this an $R$-automaton.

The LT, LTT and PT language classes have the above properties and thus such related canonical automaton.

The set of all such $R$-automata, varying in the final state set, is isomprophic to $\Lang(R)$. We have
\[ \Lang(R) = \Set{L(\A_R(F))}{F \subseteq S_R} =: \Lang^*(\A_R) . \]

\begin{mydef}
Analogously for $\omega$, we get the set of $R$-E-automata with the $\omega$-language-class
\[ \Lang^\omega_E(\A_R) := \Set{L^\omega(\A^E_R(F))}{F \subseteq S_R} , \]
$R$-Büchi-automata and
\[ \Lang^\omega_{\text{Büchi}} (\A_R) := \Set{L^\omega(\A_R^{\text{Büchi}}(F))}{F \subseteq S_R} , \]
$R$-Muller-automata and
\[ \Lang^\omega_{\text{Muller}} (\A_R) := \Set{L^\omega(\A_R^{\text{Muller}}(\F))}{\F \subseteq 2^{S_R}} . \]
\end{mydef}

%S307.1
\begin{mydef}
For a relation $R$ on $\Sigma^*$, there are various ways to construct a relation on $\Sigma^\omega$. For now, we mainly study $R^\omega := \dext R$, i.e.
\[ (\alpha,\beta) \in R^\omega \ :\Leftrightarrow \ \forall n \colon (\alpha[0,n],\beta[0,n]) \in R . \]

Analogously to $\Lang(R)$, define the $\omega$-language-class
\[ \Lang^\omega(R^\omega) := \Set{L^\omega \subseteq \Sigma^\omega}{\text{$L^\omega$ is finite union of $R^\omega$-equivalence-classes}} . \]
\end{mydef}

\subsubsection{Classification}

With this preparation, we show for some $R$ the equalities:
\begin{itemize}
\item $\Lang^\omega_E(\A_R) = \ext \Lang(R)$ (\ref{gen:lang_omega_e})
\item $\Lang^\omega_{\text{Büchi}}(\A_R) = \lim \Lang(R)$ (\ref{gen:lang_omega_buechi})
\item $\Lang^\omega_{\text{Muller}}(\A_R) = \BC \lim \Lang(R)$ (\ref{gen:lang_omega_muller})
\item $\Lang^\omega(R^\omega) = \BC \ext \Lang(R)$
\item $\BC \lim \Lang(R) \cap \ext \Langreg = \ext \Lang(R)$
\item $\BC \lim \Lang(R) \cap \lim \Langreg = \lim \Lang(R)$
\item $\lim \Lang(R) \cap \dlim \Lang(R) = \BC \ext \Lang(R)$
\end{itemize}

We will see that all those equations hold for $\Lang(LT)$, $\Lang(LTT)$ and $\Lang(PT)$.

\begin{lemma}
%\subsection{$\Lang^\omega_E(\A_R) = \ext \Lang(R)$}
%S307.2,S307.3
\label{gen:lang_omega_e}
\[ \Lang^\omega_E(\A_R) = \ext \Lang(R) \]
\begin{proof}
Let $L = \bigcup_i \left< w_i\right>_R, L \in \Lang(R)$. Then
\begin{align*}
& L^\omega = \ext L \\
\Leftrightarrow \ & L^\omega = \Set{\alpha \in \Sigma^\omega}{\exists n \colon \alpha[0,n] \in \bigcup_i \left< w_i\right>_R} \\
\Leftrightarrow \ & L^\omega = \Set{\alpha \in \Sigma^\omega}{\exists n \colon \delta_{\A_R}(\alpha[0,n]) \in \Set{\left< w_i\right>_R \subseteq S_R}{i}} \\
\Leftrightarrow \ & L^\omega = L^\omega(\A^E_R(\Set{\left< w_i\right>_R \subseteq S_R}{i}))
\end{align*}
\end{proof}
\end{lemma}

\begin{lemma}
%\subsection{$\Lang^\omega_{\text{Büchi}}(\A_R) = \lim \Lang(R)$}
%S307.2,S307.3
\label{gen:lang_omega_buechi}
\[ \Lang^\omega_{\text{Büchi}}(\A_R) = \lim \Lang(R) \]
\begin{proof}
Let $L = \bigcup_i \left< w_i\right>_R, L \in \Lang(R)$. Then
\begin{align*}
& L^\omega = \lim L \\
\Leftrightarrow \ & L^\omega = \Set{\alpha \in \Sigma^\omega}{\exists^\infty n \colon \alpha[0,n] \in \bigcup_i \left< w_i\right>_R} \\
\Leftrightarrow \ & L^\omega = \Set{\alpha \in \Sigma^\omega}{\exists^\infty n \colon \delta_{\A_R}(\alpha[0,n]) \in \Set{\left< w_i\right>_R \subseteq S_R}{i}} \\
\Leftrightarrow \ & L^\omega = L^\omega(\A^{\text{Büchi}}_R(\Set{\left< w_i\right>_R \subseteq S_R}{i}))
\end{align*}
\end{proof}
\end{lemma}

\begin{lemma}
%\subsection{$\Lang^\omega_{\text{Muller}}(\A_R) = \BC \lim \Lang(R)$}
%S307.2,S307.3
\label{gen:lang_omega_muller}
\[ \Lang^\omega_{\text{Muller}}(\A_R) = \BC \lim \Lang(R) \]
\begin{proof}
Any $L^\omega \in \BC \lim \Lang(R)$ can be described by $\BC 2^{S_R}$. $2^{2^{S_R}}$ is also finite. Thus, any $A \in \BC 2^{S_R}$ can be represented in $2^{2^{S_R}}$. This is exactly an acceptance condition in Muller.
\end{proof}
\end{lemma}

\begin{lemma}
%\subsection{$\Lang^\omega(R^\omega) = \BC \ext \Lang(R)$}
%S307.4,S307.5
\[ \Lang^\omega(R^\omega) = \BC \ext \Lang(R) \]
\begin{proof}
TODO...
\end{proof}
\end{lemma}

\begin{lemma}
%\subsection{$\BC \lim \Lang(R) \cap \ext \Langreg = \ext \Lang(R)$}
%S307.6,S307.7,S307.8
\label{gen:R-bclim-cap-ext}
\[ \BC \lim \Lang(R) \cap \ext \Langreg = \ext \Lang(R) \]

\begin{proof}
We have $\ext \Lang(R) \subseteq \ext \Langreg$ and $\ext \Lang(R) \subseteq \BC \lim \Lang(R)$. Thus, "$\supseteq$" is shown.

Now, we show "$\subseteq$". Let $L^\omega \in \BC \lim \Lang(R) \cap \ext \Langreg$. Because $L^\omega \in \ext \Langreg$, there is an E-automaton $\A^E$ which accepts $L^\omega$. We can assume that $\A^E$ is deterministic (with \ref{gen:e-determinism}).

We must find an $R$-E-automaton which accepts $L^\omega$. We will call it the $\overline{\A}^M$ E-automaton and will construct it in the following.

Let $\A^M$ be the deterministic $R$-Muller-automaton for $L^\omega$ (according to \ref{gen:R-automata} and \ref{gen:lang_omega_muller}). Without restriction, there are no final state sets in $\A^M$ which are not loops. Then, $\overline{\A}^M$ has the same states and transitions as $\A^M$.

Look at a final state $q^E$ of $\A^E$. Without restriction, we can assume that there is no path that we can reach multiple final states at once. Let $L_{q^E}$ be all words which reach $q^E$ exactly once at the end.

Let $w \in L_{q^E}$.
%In $\A^M$, after $w$, we reached a state where anything that follows will eventually reach a SCC where any possible looping subset is an element of the final state set of $\A^M$ and there is no way out of the SCC.
%Look at some SCC $S$ in $\A^M$. Let $q \in S$. Let $\F$ be the subset of the finite state set of $\A^M$ so that $q \in F$ for all $F \in \F$, i.e. $F \cap S \neq \emptyset$. We can ignore all $F$ which are not a loop because they would not accept anything because they cannot be reached infinitely often. Because $S$ is a SCC and $F$ is a loop, we also get $F \subseteq S$.
Let $q$ be the state in $\A^M$ which is reached after $w$. Let $S$ be the set of states in $\A^M$ which can be reached from $q$.
%Let $\F$ be the subset of the finite state set of $\A^M$ so that $F \cap S \neq \emptyset$ for all $F \in \F$. We also get $F \subseteq S$.

Then, $\A^M$ accepts all words in $L_q \cdot L_{q,S}^\omega$, where $L_q$ is the set of words to $q$ and $L^\omega_{q,S}$ is the set of words of possible infinite postfixes after $q$ in $S$ so that they are accepted.
%loops $q\rightarrow q$ in $\F$.
Any word with a prefix in $L_q$, which is not in $L_q \cdot L^\omega_{q,S}$, will not be accepted by $\A^M$ because $\A^M$ is deterministic. Also, because $L_{q^E} \cap L_q \neq \emptyset$ and $L_{q^E} \cdot \Sigma^\omega \subseteq L^\omega$ and $L_q \cdot L^\omega_{q,S} \subseteq L^\omega$, we get $L^\omega_{q,S} \neq \emptyset$.

Assuming $L^\omega_{q,S} \neq \Sigma^\omega$. Then we would have $L^\omega \not\in \ext \Langreg$, which is a contradiction. I.e. $L^\omega_{q,S} = \Sigma^\omega$.

Thus, $\A^M$ accepts all words in $L_q \cdot \Sigma^\omega$. Mark $q$ as a final state in $\overline{\A}^M$. Thus, $\overline{\A}^M$ E-accepts all words in $L_q \cdot \Sigma^\omega \subseteq L^\omega$.

%For any state $\tilde q$ in $\A^M$ which is not marked as a final state in the previously described way, all states in $\A^E$ after words in $L_{\tilde q}$ are not final states in $\A^E$. Thus, all words not in $L^\omega$ are not accepted by $\overline{\A}^M$.

Because we did this for all final states in $\A^E$, there is no $\alpha \in L^\omega$ which is not accepted by $\overline{\A}^M$. I.e., the $R$-E-automata $\overline{\A}^M$ accepts exactly $L^\omega$. I.e. $L^\omega \in \ext \Lang(R)$.
\end{proof}
\end{lemma}

%\subsection{$\BC \lim \Lang(R) \cap \lim \Langreg = \lim \Lang(R)$}
%S307.6,S307.8
\begin{lemma}
\[ \BC \lim \Lang(R) \cap \lim \Langreg = \lim \Lang(R) \]

\begin{proof}
This proof is loosely analogue to the proof in \ref{gen:R-bclim-cap-ext}.

We have $\lim \Lang(R) \subseteq \lim \Langreg$ and $\lim \Lang(R) \subseteq \BC \lim \Lang(R)$. Thus, "$\supseteq$" is shown.

Now, we show "$\subseteq$". Let $L^\omega \in \BC \lim \Lang(R) \cap \lim \Langreg$. Because $L^\omega \in \lim \Langreg$, there is an Büchi-automaton $\A^B$ which accepts $L^\omega$. We can assume that $\A^B$ is deterministic (with \ref{gen:e-determinism}).

We must find an $R$-Büchi-automaton which accepts $L^\omega$. We will call it the $\overline{\A}^M$ Büchi-automaton and will construct it in the following.

Let $\A^M$ be the deterministic $R$-Muller-automaton for $L^\omega$ (according to \ref{gen:R-automata} and \ref{gen:lang_omega_muller}). Without restriction, there are no final state sets in $\A^M$ which are not loops. Then, $\overline{\A}^M$ has the same states and transitions as $\A^M$.

Look at the SCC $S$ in $\A^M$. Let $q \in S$. Let $\mathcal F_q \subseteq 2^S$ be the set of final states in $\A^M$ with $q \in F$ for all $F \in \F_q$. Let $\mathcal S_q \subseteq 2^S$ be the set of loops in $S$ which include $q$.

Case 1: $\F_q \neq \mathcal S_q$.

Case 2: $\F_q = \mathcal S_q$.
In that case, mark $q$ as a final state in $\overline{\A}^M$.

For the constructed Büchi-automaton $\overline{\A}^M$, we show that it accepts exactly $L^\omega$.

Let $\alpha \in L^\omega(\overline{\A}^M)$. Let $q$ be some final state in $\overline{\A}^M$ which is infinitely often visited by $\alpha$. Then, $\F_q = \mathcal S_q$ from the construction. I.e., no matter what loops through $q$ of the related SCC are visited infinitely often by $\alpha$, it will be accepted by $\A^M$. Thus, $\alpha \in L^\omega$.

Let $\alpha \in L^\omega$. Then, the set of states $F$ infinitely often visited by $\alpha$ in $\A^M$ is some final state set of the Muller-automaton $\A^M$. In $\A^B$, there is a final state $\tilde q$ infinitely often visited by $\alpha$. Let $\alpha =: \prod_{i=1}^{\infty} w_i$ so that $\prod_{i=1}^{n} w_i$ ends up in $\tilde q$ in $\A^B$ for all $n \in \N$ for shortest possible $w_i$ (i.e. we don't miss any $\tilde q$). Let $S$ be the SCC in $\A^M$ where we finally end up with $\alpha$. Then, $F \subseteq S$.

There must be a $q \in F$ so that $\F_q = \mathcal S_q$. Then, by construction of $\overline{\A}^M$, $q$ is a final state in $\overline{\A}^M$ and thus, $\alpha \in L^\omega(\overline{\A}^M)$.

Let us show that there is such $q \in F$ by contradiction. I.e. assume there is no such $q \in F$. I.e. for all $q \in F$, $\F_q \neq \mathcal S_q$. Of course we have $F \in \F_q$ for all $q \in F$.

Let $\mathcal P_{\tilde q}$ be the set of loops in $\A^M$ so that all words which end up looping there infinitely would also visit $\tilde q$ infinitely often in $\A^B$. Of course, all $P \in \mathcal P_{\tilde q}$ will be final state sets in $\A^M$ because $\A^B$ would accept. Define $\mathcal P_{\tilde q,S} := \Set{P \in \mathcal P_{\tilde q}}{P \subseteq S}$. No matter how much other infinte loops in $S$ we add to $\alpha$ so that we still visit some loops from $\mathcal P_{\tilde q,S}$ infinitely often, $\A^M$ and $\A^B$ will keep accepting. Thus, for $P \in \mathcal P_{\tilde q,S}$, every $P' \supseteq P$, $P' \in \mathcal S_S$, we have $P' \in \mathcal P_{\tilde q,S}$.

...

%For any $q \in F$, we have $\F_q \neq \emptyset$. Thus, $q$ is marked as a final state in $\overline{\A}^M$ and thus, $\overline{\A}^M$ accepts $\alpha$.

%Let us show $\F_q = \mathcal S_q$ under the condition $\F_q \neq \emptyset$. Of course we have $\F_q \subseteq \mathcal S_q$. Let $\A^B$ be the deterministic Büchi automaton for $L^\omega$ (we have $L^\omega \in \lim \Langreg$).
%Let $L_q$ be the set of words which reach $q$ in $\A^M$.
%We show the assumption by contradiction: Let $S \in \mathcal S_q$ with $S \not\in \F_q$. Let $F \in \F_q$ with $F \subseteq S$ or $F \cup S \in \F_q$ (TODO: does that exists?! -> no!).

%Let $L_{q,S} \subseteq \Sigma^*$ be the set of non-looping words from $q \rightarrow q$ in $S$, and likewise $L_{q,F} \subseteq \Sigma^*$ in $F$.

%Let $w \in L_q$, $\overline w_1 \in L_{q,F}$. Then, the infinite states visited by $w \overline w_1$ are exactly $F$. Thus, we have $w \overline w_1 \in L^\omega$.
%Let $\overline w_2 \in L_{q,S}$.
%In $\A^B$, after $w \cdot \Set{\overline w_1, \overline w_2}^\omega$, we will eventually reach a final SCC $S_B$. Let $\tilde w \in w \cdot \Set{\overline w_1, \overline w_2}^*$, so that $\tilde w$ reaches $S_B$. We have $\tilde w \cdot \Set{\overline w_1,\overline w_2}^* \cdot \overline w_1^\omega \subseteq L^\omega$.

%Let $W := \tilde w \cdot \Set{\overline w_1,\overline w_2}^*$, $W_1 := \overline w_1^+$, $W_2 := (\overline w_1^* \overline w_2 \overline w_1^*)^+$.
%Search $q_B \in S_B$ with $W \rightarrow q_B$, $q_B \xrightarrow{W_1} q_B$ and $q_B \xrightarrow{W_2} q_B$. Assuming such a $q_B$ exists with $\tilde w_0 \in W$, $\tilde w_1 \in W_1$, $\tilde w_2 \in W_2$ so that $\tilde w_0 \rightarrow q_B$, $q_B \xrightarrow{\tilde w_i} q_B$ for $i \in \Set{1,2}$. Because $\tilde w_0 \cdot \tilde w_1^\omega \in L^\omega$, there is a final state $\tilde q_B$ in $\A^B$ on the path $q_B \xrightarrow{\tilde w_1} q_B$. Then, $\beta := \tilde w_0 \cdot (\tilde w_1 \cdot \tilde w_2)^\omega$ also visits $\tilde q_B$ infinitely often, thus $\A^B$ also accepts $\beta$. In $\A^M$, $\beta$ visits exactly $F \cup S$ infinitely often. Thus, $\A_M$ does not accept $\beta$. That is a contradiction. Thus, $\F_q = \mathcal S_q$.

%TODO: Now show that such $q_B$ exists with $\tilde w_0 \in W$, $\tilde w_1 \in W_1$, $\tilde w_2 \in W_2$.
\end{proof}
\end{lemma}

\begin{lemma}
%\subsection{$\lim \Lang(R) \cap \dlim \Lang(R) = \BC \ext \Lang(R)$}
\[ \lim \Lang(R) \cap \dlim \Lang(R) = \BC \ext \Lang(R) \]
\begin{proof}
\end{proof}
\end{lemma}



