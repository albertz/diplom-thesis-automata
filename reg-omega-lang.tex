\section{regular $\omega$-languages}

The class of regular $\omega$-languages can be defined in many different ways. We will use one common definition and show some equivalent descriptions.
\[ \Lang^\omega(reg) := \Set{ \cup_i U_i \cdot V_i^\omega }{ U_i, V_i \in \Lang^*(reg) } \]

A different, very common description is in terms of automata.

An automaton $\A = (Q, \Sigma, E, I, F)$ \defword{Büchi-accepts} a word $\alpha = (a_0,a_1,a_2,...) \in \Sigma^\omega$ iff there is an infinite run $q_0 \rightarrow^{a_0} q_1 \rightarrow^{a_1} q_2 \rightarrow^{a_2} q_3 ...$ with $q_0 \in I$ and $\{ q_i | q_i \in F \}$ infinite, i.e. which reaches a state in $F$ infinitely often.

The language $L^\omega(\A)$ is defined as the set of all infinite words which are Büchi-accepted by  $\A$.

An automaton $\A$ is a Büchi automaton iff we use the Büchi-acceptence.

The set of all languages accepted by a non-deterministic Büchi automaton is exactly $\Lang^\omega(reg)$. (S218,R101) Deterministic Büchi automata are less powerful, e.g. they cannot recognise $(a+b)^* b^\omega$.

There are some different forms of $\omega$-automata, e.g. the Rabin automata and the Muller automata. We see that the class of languages accepted by non-deterministic Büchi automata is equal to deterministic Rabin automata and deterministic Muller automata. (S407)

We also see that this is equal to boolean combinations of languages accepted by deterministic Büchi automata. Under this regard, an operator of interest is $\lim(L) := \Set{ \alpha \in \Sigma^\omega }{ \exists^\omega n \colon \alpha[0,n] \in L }$. We see that $\lim(\Lang^\omega)$ is equal to the languages accepted by deterministic Büchi automata. (S407) Thus:
\[  \BC \lim \Lang^*(reg) = \Lang^\omega(reg) \]

Some other descriptions:
$\Lang^\omega(reg) = \Set{ \cup_i U_i \cdot \lim V_i }{ U_i, V_i \in \Lang^*(reg) }$ (S218,S411,R107)
$\Lang^\omega(reg) = \Set{A \subset \Sigma^\omega}{A \text{definable in} L_2(\Sigma)}$

We will formulate some properties of interest in a general form for a $*$-language class $\Lang$ which all hold for $\Lang^*(reg)$. Let $L,A,B \in \Lang$.
\begin{itemize}
\item E1: $L \cdot \Sigma^* \in \Lang$ (not suffix sensitive)
\item E2a: $A \cup B \in \Lang$
\item E2b: $A \cap B \in \Lang$
\item E3: $- L \in \Lang$ (closed under complementation) (S303.E3, S218, R101)
\end{itemize}
