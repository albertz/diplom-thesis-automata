%!TEX root =  index.tex

% some useful stuff:
% http://www.automata.rwth-aachen.de/material/skripte/latex/latex.pdf
% http://en.wikibooks.org/wiki/LaTeX/Mathematics
% http://en.wikibooks.org/wiki/LaTeX/Advanced_Mathematics
% http://en.wikibooks.org/wiki/LaTeX/Theorems

\newtheorem{thm}{Theorem}[section]
\newtheorem{lem}[thm]{Lemma}

\theoremstyle{definition}\newtheorem{mydef}{Definition}[section]
\theoremstyle{plain}\newtheorem{lemma}{Lemma}[section]
\theoremstyle{definition}\newtheorem{algo}{Algorithm}[section]
\newtheorem{theorem}{Theorem}[section]

\newcommand{\K}{\mathcal{K}}
\newcommand{\Reg}{\text{Reg}}
\newcommand{\Lang}{\mathcal{L}}
\newcommand{\A}{\mathcal{A}}
\newcommand{\T}{\mathcal{T}}
\newcommand{\F}{\mathcal{F}}
\newcommand{\Q}{\mathbb{Q}}
\newcommand{\R}{\mathbb{R}}
\newcommand{\N}{\mathbb{N}}
\newcommand{\B}{\mathbb{B}}
\newcommand{\Power}{\mathcal{P}}

\newcommand{\mathtext}[1]{\textup{\textrm{#1}}}
\newcommand{\PT}{\mathtext{PT}}
\newcommand{\LT}{\mathtext{LT}}

% got some help here: http://tex.stackexchange.com/questions/13554/define-something-like-lim-but-for-another-name

\newcommand{\ext}{\operatorname{ext}}
\newcommand{\Inf}{\operatorname{Inf}}
\newcommand{\BC}{\operatorname{BC}}
\newcommand{\dext}{\operatorname{\overline{ext}}}
\newcommand{\dlim}{\operatorname{\overline{lim}}}
\newcommand{\Kleene}{\operatorname{\widehat{Kleene}}}
\newcommand{\limClosure}{\operatorname{\widehat{lim}}}
\newcommand{\Occ}{\operatorname{Occ}}
\newcommand{\existsinf}{\exists^\omega}
\newcommand{\overx}{\overset{\times}}
%\newcommand{\overx}{\stackrel{\times}}
\newcommand{\Ax}{\overx{\A}}
\newcommand{\Langreg}{\Lang^*(\text{reg})}
\newcommand{\LangOreg}{\Lang^\omega(\text{reg})}

\newcommand{\defword}[1]{{\bf #1}}

% inspired by http://ftp.fernuni-hagen.de/ftp-dir/pub/mirrors/www.ctan.org/macros/latex/contrib/braket/braket.sty
\def\mid@vertical{\mskip1mu\vrule\mskip1mu}
\def\midvert{\egroup\;\mid@vertical\;\bgroup}
\NewDocumentCommand\Set{mg}{%
    \IfNoValueTF{#2}{%
        \ensuremath{\left\{ #1 \right\}}%
    }{%
        \ensuremath{\left\{ {#1} \;\mid@vertical\; {#2} \right\}}%
    }%
}

%\newcommand{\SetS}[1]{\bigl\{ #1 \bigr\}}
%\newcommand{\SetC}[2]{\bigl\{ #1 \bigm| #2 \bigr\}}
%\DeclarePairedDelimiterX\SetC[2]{\lbrace}{\rbrace}{ #1 \,\delimsize|\, #2 }

%\newcommand{\abs}[1]{\mathopen| #1 \mathclose|}
%\newcommand{\Abs}[1]{\left| #1 \right|}
\newcommand{\abs}[1]{\left| #1 \right|}

% http://de.wikibooks.org/wiki/LaTeX-W%C3%B6rterbuch:_today
\def\monthgerman{\ifcase\month \or
  Januar\or Februar\or M\"arz\or April\or Mai\or Juni\or
  Juli\or August\or September\or Oktober\or November\or Dezember\fi}
\def\todaygerman{\number\day.~\monthgerman\space\number\year}
