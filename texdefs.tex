%!TEX root =  index.tex

% some useful stuff:
% http://www.automata.rwth-aachen.de/material/skripte/latex/latex.pdf
% http://en.wikibooks.org/wiki/LaTeX/Mathematics
% http://en.wikibooks.org/wiki/LaTeX/Advanced_Mathematics
% http://en.wikibooks.org/wiki/LaTeX/Theorems

\newtheorem{thm}{Theorem}[section]
\newtheorem{lem}[thm]{Lemma}

\newcommand{\K}{\mathcal{K}}
\newcommand{\Reg}{\text{Reg}}
\newcommand{\Lang}{\mathcal{L}}
\newcommand{\A}{\mathcal{A}}
\newcommand{\Q}{\mathcal{Q}}
\newcommand{\T}{\mathcal{T}}
\newcommand{\R}{\mathcal{R}}

\newcommand{\PT}{\textup{\textrm{piece-wise testable}}}

% got some help here: http://tex.stackexchange.com/questions/13554/define-something-like-lim-but-for-another-name

\newcommand{\ext}{\operatorname{ext}}
\newcommand{\Inf}{\operatorname{Inf}}
\newcommand{\BC}{\operatorname{BC}}

\newcommand{\defword}[1]{{\bf #1}}

\NewDocumentCommand\Set{mg}{%
    \IfNoValueTF{#2}{%
        \ensuremath{\bigl\{ #1 \bigr\}}%
    }{%
        \ensuremath{\bigl\{ #1 \bigm| #2 \bigr\}}%
    }%
}

%\newcommand{\SetS}[1]{\bigl\{ #1 \bigr\}}
%\newcommand{\SetC}[2]{\bigl\{ #1 \bigm| #2 \bigr\}}
%\DeclarePairedDelimiterX\SetC[2]{\lbrace}{\rbrace}{ #1 \,\delimsize|\, #2 }

